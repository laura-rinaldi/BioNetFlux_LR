\documentclass[11pt,a4paper]{article}
\usepackage[utf8]{inputenc}
\usepackage[english]{babel}
\usepackage{amsmath,amsfonts,amssymb}
\usepackage{graphicx}
\usepackage{geometry}
\usepackage{fancyhdr}
\usepackage{listings}
\usepackage[dvipsnames]{xcolor}
\usepackage{hyperref}
\usepackage{tocloft}
\usepackage{titlesec}
\usepackage{float}
\usepackage{booktabs}
\usepackage{array}
\usepackage{longtable}
\usepackage{tikz}
\usepackage{bclogo}
\usepackage{xcolor}
\usepackage{tcolorbox}
\usepackage{enumitem}

% Define a custom TODO box
\newtcolorbox{todobox}{
	colback=yellow!10!white,
	colframe=orange!70!black,
	title=\textbf{TODO},
	fonttitle=\bfseries,
	coltitle=white,
	boxrule=0.8pt,
	arc=4pt,
	left=6pt,
	right=6pt,
	top=6pt,
	bottom=6pt
}

\newtcolorbox{warningbox}{
	colback=red!5!white,
	colframe=red!75!black,
	title=\textbf{WARNING!},
	fonttitle=\bfseries,
	coltitle=white,
	boxrule=1pt,
	arc=4pt,
	left=6pt,
	right=6pt,
	top=6pt,
	bottom=6pt
}


% Define a custom box for deprecated notices
\newtcolorbox{deprecatedbox}{
	colback=red!5!white,
	colframe=red!75!black,
	title=DEPRECATED,
	fonttitle=\bfseries,
	boxrule=0.8pt,
	arc=4pt,
	left=1mm,
	right=1mm,
	top=1mm,
	bottom=1mm
}


% Define a custom box for deprecated notices
\newtcolorbox{structurebox}{
	colback=teal!5!white,
	colframe=teal!75!black,
	title=DATA STRUCTURE,
	fonttitle=\bfseries,
	boxrule=0.8pt,
	arc=4pt,
	left=1mm,
	right=1mm,
	top=1mm,
	bottom=1mm
}



\newtcolorbox{notebox}{
	colback=olive!5!white,
	colframe=olive!75!black,
	title=NOTE,
	fonttitle=\bfseries,
	boxrule=0.8pt,
	arc=4pt,
	left=1mm,
	right=1mm,
	top=1mm,
	bottom=1mm
}

	

% Page setup
\geometry{margin=2.5cm}
\pagestyle{fancy}
\fancyhf{}
\fancyhead[L]{\textsc{BioNetFlux Documentation}}
\fancyhead[R]{\thepage}
\fancyfoot[C]{\textit{Multi-Domain Biological Network Flow Simulation}}

% Hyperlink setup
\hypersetup{
    colorlinks=true,
    linkcolor=blue,
    filecolor=magenta,      
    urlcolor=cyan,
    pdftitle={BioNetFlux Documentation},
    pdfauthor={BioNetFlux Development Team},
}

% Code listing setup
\definecolor{codegreen}{rgb}{0,0.6,0}
\definecolor{codegray}{rgb}{0.5,0.5,0.5}
\definecolor{codepurple}{rgb}{0.58,0,0.82}
\definecolor{backcolour}{rgb}{0.95,0.95,0.92}

\lstdefinestyle{mystyle}{
    backgroundcolor=\color{backcolour},   
    commentstyle=\color{codegreen},
    keywordstyle=\color{magenta},
    numberstyle=\tiny\color{codegray},
    stringstyle=\color{codepurple},
    basicstyle=\ttfamily\footnotesize,
    breakatwhitespace=false,         
    breaklines=true,                 
    captionpos=b,                    
    keepspaces=true,                 
    numbers=left,                    
    numbersep=5pt,                  
    showspaces=false,                
    showstringspaces=false,
    showtabs=false,                  
    tabsize=2
}

\lstset{style=mystyle}

% Custom commands
\newcommand{\code}[1]{\texttt{#1}}
\newcommand{\bionetflux}{\textsc{BioNetFlux}}

\newcommand{\brokencode}{\bcattention}

% Title page customization
\title{\Huge {\textbf{\bionetflux{} Documentation}} \\[0.5cm]
       \Large Multi-Domain Biological Network Flow Simulation}
\author{BioNetFlux Development Team}
\date{\today}

\begin{document}

% Title page
\begin{titlepage}
    \centering
    
    % BioNetFlux Logo
    \includegraphics[width=0.6\textwidth]{BioNetFlux_Logo.jpg}\\[1cm]
    
    {\Huge \textbf{\bionetflux{}} \\[0.5cm]}
    {\Large \textbf{Documentation} \\[1cm]}
    
    {\large Multi-Domain Biological Network Flow Simulation \\[0.5cm]}
    {\large A Python Framework for Complex Network Geometries \\[2cm]}

    
    {\Large BioNetFlux Development Team \\[0.5cm]}
    {\large \today}
    
    \vfill
    
    {\footnotesize 
    \textit{Comprehensive guide to multi-domain biological transport simulations} \\
    \textit{including Keller-Segel chemotaxis and organ-on-chip modeling}
    }
        
        \vskip3cm
        
    % Barra bar
    \includegraphics[width=\textwidth]{Barra_D34Health.png}\\[2cm]
\end{titlepage}

% Table of contents
\tableofcontents

\clearpage

\section*{Acknowledgements}
\addcontentsline{toc}{section}{Acknowledgements}

The development of BioNetFlux was carried out thanks to the support by the Italian Ministry of Research, under the complementary actions to the NRRP ``D34Health - Digital Driven Diagnostics, prognostics and therapeutics for sustainable Health care" Grant (\#PNC0000001).

\

Portions of this documentation were drafted with assistance from the Claude Sonnet 4 AI language model, with all content reviewed and edited by the authors.

\

AI tools were used as programming and writing aids only. All mathematical derivations,
algorithmic design, and numerical validation were performed by the authors.
\clearpage

\section{Introduction}

\bionetflux{} is a computational framework designed for simulating biological transport phenomena on complex one dimensional networks. Based on an Hybridized Discontinuous Galerkin (HDG) approach, the framework specializes in solving coupled partial differential equations (PDEs) on multi-arc (branch/channel) networks, with particular focus on:

\begin{itemize}
    \item \textbf{Keller-Segel chemotaxis models}: Cell migration driven by chemical gradients
    \item \textbf{Organ-on-Chip systems}: Microfluidic device simulations with multiple compartments
    \item \textbf{Multi-arc networks}: Complex geometries with different type of junction conditions and interface constraints
\end{itemize}

\subsection{Key Features}

\begin{itemize}
    \item \textbf{Multi-Arc Support}: Handle complex network topologies with arbitrary arc connections
    \item \textbf{Arbitrary equation number}: Parametric handling of the number of equations per arc
    \item \textbf{Geometry Management}: Intuitive geometry definition using the \code{DomainGeometry} class
    \item \textbf{Flexible Constraints}: Support for Neumann, Dirichlet, and Robin boundary conditions and  Kedem-Katchalsky junction conditions
    \item \textbf{Advanced Visualization}: 2D curve plots, 3D flat views, and bird's eye network visualization
    \item \textbf{Time Evolution}: Euler implicit time stepping with nonlinear solver
    \item \textbf{Static Condensation}: Efficient element-level solution elimination
   \item \textbf{Extensibility}: Adding new problem classes by writing problem specific static condensation modules 
\end{itemize}


\begin{warningbox}
	Throughout the code we use the word {\tt Domain} as a synonym of {\tt Arc}
\end{warningbox}

\section{Architecture Overview}

The \bionetflux{} framework is organized into several interconnected modules following a clean, modular architecture:

\begin{lstlisting}[language=bash, caption={BioNetFlux Directory Structure}]
BioNetFlux/
├── src/
│   └── bionetflux/
│       ├── core/                    # Core mathematical components
│       │   ├── problem.py           # Problem definitions and parameters
│       │   ├── discretization.py    # Spatial and temporal discretization
│       │   ├── constraints.py       # Boundary and interface conditions
│       │   ├── bulk_data.py         # Element-level data management
│       │   ├── lean_bulk_data_manager.py  # Memory-efficient data handling
│       │   ├── lean_global_assembly.py   # Global system assembly
│       │   ├── static_condensation_*.py  # Problem-specific condensation
│       │   └── flux_jump.py         # Interface flux calculations
│       ├── geometry/                # Geometry management
│       │   └── domain_geometry.py   # Network topology and layout
│       ├── problems/                # Problem definitions
│       │   ├── test_problem.py      # Reference test problems
│       │   ├── test_problem2.py     # Alternative test configurations
│       │   └── ooc_test_problem.py  # OrganOnChip examples
│       ├── utils/                   # Utility modules
│       │   └── elementary_matrices.py  # Finite element matrices
│       └── visualization/           # Plotting and visualization
│           └── lean_matplotlib_plotter.py  # Multi-mode plotting
├── tests/                          # Pytest test suite
│   ├── test_geometry.py            # Geometry module tests
│   ├── test_problem.py             # Problem class tests
│   ├── test_lean_bulk_data_manager.py  # Data manager tests
│   ├── test_lean_global_assembly.py    # Assembly tests
│   ├── test_static_condensation_setup.py  # SC setup tests
│   └── test_lean_setup.py          # Complete setup tests
├── setup_solver.py                 # Main solver setup interface
├── matlab_reference/               # MATLAB reference implementations
│   ├── TestProblem.m               # MATLAB test problem (4-eq OoC)
│   ├── EmptyProblem.m              # MATLAB empty template
│   ├── StaticC.m                   # Static condensation reference
│   └── scBlocks.m                  # Matrix building blocks
└── docs/                          # Documentation
\end{lstlisting}

\subsection{Core Architecture Principles}

\begin{enumerate}
    \item \textbf{Lean Design}: Minimal data storage with on-demand component creation
    \item \textbf{Parameter Passing}: Framework objects passed as parameters rather than stored
    \item \textbf{Component Caching}: Expensive objects created once and cached for efficiency
    \item \textbf{MATLAB Reference Alignment}: Python implementations follow MATLAB reference structure
    \item \textbf{Modular Problem Types}: Static condensation implementations for different physics
    \item \textbf{Comprehensive Testing}: Pytest-compatible test suite with fixtures and parametrization
\end{enumerate}

\subsection{Core Components}

\begin{enumerate}
    \item \textbf{Problem Definition}: Physical parameters following MATLAB OoC structure (4-equation systems)
    \item \textbf{Geometry Management}: Network topology using DomainGeometry class
    \item \textbf{Static Condensation}: Element-level solution elimination following StaticC.m algorithm
    \item \textbf{Global Assembly}: Lean system assembly with parameter-based validation
    \item \textbf{Discretization System}: Spatial mesh and temporal discretization management
    \item \textbf{Constraint Management}: Interface conditions and boundary constraints
    \item \textbf{Visualization System}: Multi-mode plotting with 2D curves, 3D flat views, and bird's eye views
\end{enumerate}

\section{Modules}

\subsection{Core Module (\code{bionetflux.core})}

The core module contains the fundamental mathematical and computational components:

\subsubsection{Problem Class (\code{problem.py})}

The \code{Problem} class encapsulates the physics of a single domain/arc, following the MATLAB OoC parameter structure:

\begin{lstlisting}[language=Python, caption={Problem Class Structure}]
class Problem:
    def __init__(self, neq=2, domain_start=0.0, domain_length=1.0, 
                 parameters=None, problem_type="keller_segel", 
                 name="unnamed_problem"):
        # Physical domain definition and equation parameters
        # Follows MATLAB OoC structure: [nu, mu, epsilon, sigma, a, b, c, d, chi]
        
    # Parameter management (aligned with MATLAB TestProblem.m)
    def get_parameter(self, index)
    def set_parameter(self, index, value)  
    def set_parameters(self, parameters)
    
    # Function management (compatible with MATLAB u0, force functions)
    def set_initial_condition(self, eq_idx, ic_func)  # Maps to problem.u0{i}
    def set_force(self, eq_idx, force_func)           # Maps to force{i}
    def set_solution(self, eq_idx, sol_func)
    def set_chemotaxis(self, chi_func, dchi_func)     # Maps to lambda, dlambda
    
    # Validation and testing
    def validate_problem(self, verbose=False)
    def test_functions(self, verbose=False)
    def run_self_test(self, verbose=False)
\end{lstlisting}

\textbf{MATLAB Parameter Mapping:}
\begin{itemize}
    \item \code{parameters[0:3]}: Viscosity parameters (\code{nu, mu, epsilon, sigma})
    \item \code{parameters[4:7]}: Reaction parameters (\code{a, b, c, d})
    \item \code{parameters[8]}: Coupling parameter (\code{chi})
    \item \code{lambda}: Nonlinear function (set via \code{set\_chemotaxis})
\end{itemize}

\subsubsection{Static Condensation System}

Problem-specific static condensation implementations following the MATLAB StaticC.m algorithm:

\begin{lstlisting}[language=Python, caption={Static Condensation Architecture}]
# Factory pattern for different problem types
StaticCondensationFactory.create(problem, global_disc, 
                                elementary_matrices, domain_idx)

# Available implementations (following StaticC.m structure):
# - static_condensation_keller_segel.py  # 2-equation Keller-Segel
# - static_condensation_organ_on_chip.py # 4-equation OoC (matches MATLAB)
# - static_condensation_base.py          # Base class interface

# Each implementation follows MATLAB StaticC.m pattern:
class StaticCondensationBase:
    def build_matrices()  # Pre-compute matrices (follows scBlocks.m)
    def static_condensation(local_trace, local_source)  # StaticC.m algorithm
        # Returns: (U, hJ, dhJ) matching MATLAB output structure
\end{lstlisting}

\textbf{MATLAB Alignment:}
\begin{itemize}
    \item \code{build\_matrices()}: Implements scBlocks.m matrix construction
    \item \code{static\_condensation()}: Follows StaticC.m algorithm exactly
    \item Matrix naming: Uses MATLAB convention (L1, B1, C2, etc.)
    \item Input/Output: \code{(hU, rhs)} → \code{(U, hJ, dhJ)}
\end{itemize}

\subsubsection{Lean Global Assembly (\code{lean\_global\_assembly.py})}

Memory-efficient global system assembly with parameter-passing design:

\begin{lstlisting}[language=Python, caption={Lean Global Assembly}]
class GlobalAssembler:
    # Factory method creation
    @classmethod
    def from_framework_objects(problems, global_discretization, 
                              static_condensations, constraint_manager=None)
    
    # Core assembly methods
    def assemble_residual_and_jacobian(global_solution, forcing_terms,
                                      static_condensations, time)
    def initialize_bulk_data(problems, discretizations, time)
    def compute_forcing_terms(bulk_data_list, problems, 
                             discretizations, time, dt)
    
    # Solution management
    def create_initial_guess_from_problems(problems, discretizations, time)
    def create_initial_guess_from_bulk_data(bulk_data_list)
    def get_domain_solutions(global_solution)
\end{lstlisting}

\subsubsection{Lean Bulk Data Manager (\code{lean\_bulk\_data\_manager.py})}

Efficient element-level data management with minimal storage:

\begin{lstlisting}[language=Python, caption={Lean Bulk Data Manager}]
class BulkDataManager:
    def __init__(self, domain_data_list)  # Minimal storage approach
    
    # Static factory method for domain data extraction
    @staticmethod
    def extract_domain_data_list(problems, discretizations, 
                                static_condensations)
    
    # Core operations with parameter passing
    def create_bulk_data(domain_idx, problem, discretization, dual=False)
    def initialize_all_bulk_data(problems, discretizations, time)
    def compute_forcing_terms(bulk_data_list, problems, 
                             discretizations, time, dt)
    def compute_total_mass(bulk_data_list)
    
    # Parameter validation
    def _validate_framework_objects(problems, discretizations, 
                                   static_condensations, operation_name)
\end{lstlisting}

\subsection{Geometry Module (\code{bionetflux.geometry})}

\subsubsection{DomainGeometry Class (\code{domain\_geometry.py})}

Intuitive network topology definition with comprehensive validation:

\begin{lstlisting}[language=Python, caption={DomainGeometry Class}]
class DomainGeometry:
    def __init__(self, name="unnamed_geometry")
    
    # Domain management
    def add_domain(self, extrema_start, extrema_end, 
                   domain_start=None, domain_length=None,
                   name=None, **metadata)
    def remove_domain(self, domain_id)
    def get_domain(self, domain_id)
    def find_domain_by_name(self, name)
    
    # Network analysis
    def get_connectivity_info(self)
    def find_intersections(self, tolerance=1e-6)
    def get_bounding_box(self)
    def suggest_parameter_spacing(self, gap=0.1)
    
    # Validation and testing
    def validate_geometry(self, verbose=False)
    def run_self_test(self, verbose=False)
    
    # Predefined test geometries
    @staticmethod
    def create_test_geometries()
\end{lstlisting}

\textbf{Domain Information Structure:}
\begin{lstlisting}[language=Python, caption={Domain Information Dataclass}]
@dataclass
class DomainInfo:
    domain_id: int
    extrema_start: Tuple[float, float]  # Physical coordinates
    extrema_end: Tuple[float, float]
    domain_start: float                 # Parameter space
    domain_length: float
    name: str
    metadata: Dict[str, Any]
\end{lstlisting}

\subsection{Problems Module (\code{bionetflux.problems})}

Problem-specific implementations following standardized \code{create\_global\_framework} pattern and MATLAB reference structure:

\subsubsection{Problem Module Structure}

Each problem module implements the standard pattern with MATLAB compatibility:

\begin{lstlisting}[language=Python, caption={Standard Problem Module Pattern}]
def create_global_framework():
    """
    Standard problem creation pattern (follows MATLAB TestProblem.m structure).
    
    Returns:
        Tuple: (problems, global_discretization, constraint_manager, problem_name)
    """
    # 1. Parameter Configuration (matches MATLAB parameter order)
    # MATLAB: nu, mu, epsilon, sigma, a, b, c, d, chi, lambda
    parameters = np.array([nu, mu, epsilon, sigma, a, b, c, d, chi])
    
    # 2. Function Definitions (compatible with MATLAB problem.u0{i}, force{i})
    def initial_conditions_u(s, t=0.0):  # Maps to problem.u0{1}
    def initial_conditions_omega(s, t=0.0):  # Maps to problem.u0{2}  
    def initial_conditions_v(s, t=0.0):  # Maps to problem.u0{3}
    def initial_conditions_phi(s, t=0.0):  # Maps to problem.u0{4}
    
    def lambda_func(x):  # Maps to MATLAB lambda function
    def dlambda_func(x):  # Maps to MATLAB dlambda function
    
    # 3. Domain and Discretization Setup
    # 4. Problem Instance Creation with MATLAB-compatible parameters
    # 5. Constraint Setup
    # 6. Global Discretization and Time Configuration
    
    return problems, global_discretization, constraint_manager, problem_name
\end{lstlisting}

\subsubsection{Available Problem Modules}

\begin{itemize}
    \item \textbf{test\_problem.py}: Multi-domain reference problem with comprehensive constraint testing
    \item \textbf{test\_problem2.py}: Alternative test configuration for validation
    \item \textbf{ooc\_test\_problem.py}: OrganOnChip analytical test case following MATLAB TestProblem.m structure
\end{itemize}

\subsection{Setup Interface (\code{setup\_solver.py})}

\subsubsection{SolverSetup Class}

Lean orchestration of solver components with lazy loading and caching:

\begin{lstlisting}[language=Python, caption={SolverSetup Interface}]
class SolverSetup:
    def __init__(self, problem_module="bionetflux.problems.test_problem2")
    
    # Lazy-loaded properties (cached on first access)
    @property
    def elementary_matrices(self)    # ElementaryMatrices instance
    @property 
    def static_condensations(self)   # List of SC implementations
    @property
    def global_assembler(self)       # GlobalAssembler instance
    @property
    def bulk_data_manager(self)      # BulkDataManager instance
    
    # Solution vector management
    def create_initial_conditions(self)
    def create_global_solution_vector(self, trace_solutions, multipliers)
    def extract_domain_solutions(self, global_solution)
    
    # System information and validation
    def get_problem_info(self)
    def validate_setup(self, verbose=False)

# Factory functions
def create_solver_setup(problem_module)
def quick_setup(problem_module, validate=True)
\end{lstlisting}

\subsection{Visualization Module (\code{bionetflux.visualization})}

\subsubsection{LeanMatplotlibPlotter (\code{lean\_matplotlib\_plotter.py})}

Three complementary visualization modes optimized for network analysis:

\begin{lstlisting}[language=Python, caption={Visualization System}]
class LeanMatplotlibPlotter:
    def __init__(self, problems, discretizations, 
                 equation_names=None, figsize=(12, 8))
    
    # Primary visualization modes
    def plot_2d_curves(self, trace_solutions, title=None,
                       show_mesh_points=True, save_filename=None)
    def plot_flat_3d(self, trace_solutions, equation_idx=0,
                     view_angle=(30, 45), save_filename=None)
    def plot_birdview(self, trace_solutions, equation_idx=0,
                      time=0.0, save_filename=None)
    
    # Comparison and analysis
    def plot_comparison(self, initial_traces, final_traces,
                       initial_time=0.0, final_time=1.0,
                       save_filename=None)
    
    # Utility methods
    def show_all(self)
    def close_all(self)
\end{lstlisting}

\textbf{Visualization Features:}
\begin{enumerate}
    \item \textbf{2D Curve Plots}: Domain-by-domain solution profiles with mesh points
    \item \textbf{Flat 3D View}: Network topology with solution-colored scatter points above
    \item \textbf{Bird's Eye View}: Top-down network overview with color-coded solution values
    \item \textbf{Comparison Plots}: Side-by-side initial vs. final state analysis
\end{enumerate}

\section{Getting Started}

\subsection{Installation}

\begin{enumerate}
    \item Clone the repository:
    \begin{lstlisting}[language=bash]
git clone <repository-url>
cd BioNetFlux
    \end{lstlisting}
    
    \item Set up Python path:
    \begin{lstlisting}[language=Python]
import sys
sys.path.insert(0, '/path/to/BioNetFlux/src')
    \end{lstlisting}
\end{enumerate}

\subsection{Basic Usage}

\begin{lstlisting}[language=Python, caption={Basic Usage Example}]
from setup_solver import quick_setup
from bionetflux.visualization.lean_matplotlib_plotter import LeanMatplotlibPlotter

# Load a problem (follows MATLAB TestProblem.m structure)
setup = quick_setup("bionetflux.problems.ooc_test_problem", validate=True)

# Create initial conditions
trace_solutions, multipliers = setup.create_initial_conditions()

# Initialize visualization
plotter = LeanMatplotlibPlotter(
    problems=setup.problems,
    discretizations=setup.global_discretization.spatial_discretizations
)

# Plot initial conditions (all 4 equations for OoC problems)
plotter.plot_2d_curves(trace_solutions, title="Initial Conditions")
plotter.plot_birdview(trace_solutions, equation_idx=0, time=0.0)
\end{lstlisting}

\subsection{MATLAB Reference Integration}

The Python implementation closely follows the MATLAB reference structure:

\begin{lstlisting}[language=Python, caption={MATLAB-Python Parameter Mapping}]
# MATLAB TestProblem.m parameters:
# nu = 1., mu = 2., epsilon = 1., sigma = 1.
# a = 0., b = 1., c = 0., d = 1., chi = 1.

# Python equivalent:
parameters = np.array([1.0, 2.0, 1.0, 1.0,  # nu, mu, epsilon, sigma
                      0.0, 1.0, 0.0, 1.0,   # a, b, c, d  
                      1.0])                  # chi

# MATLAB initial conditions: problem.u0{1} = @(x,t) sin(2*pi*x);
# Python equivalent:
problem.set_initial_condition(0, lambda s: np.sin(2*np.pi*s))

# MATLAB Neumann data: problem.fluxu0{1} = @(t) 0.;
# Python equivalent: handled by constraint_manager.add_neumann()
\end{lstlisting}


\section{Creating New Problems}

\subsection{Problem Structure Template}

Create a new file in \code{ooc1d/problems/} following this structure:

\begin{lstlisting}[language=Python, caption={Problem Template Structure}]
# File: ooc1d/problems/my_new_problem.py
import numpy as np
from ..core.problem import Problem
from ..core.discretization import Discretization, GlobalDiscretization
from ..core.constraints import ConstraintManager
from ..geometry import DomainGeometry

def create_global_framework():
    """
    Create a new multi-domain problem.
    Returns: problems, global_discretization, 
             constraint_manager, problem_name
    """
    # 1. Global parameters
    neq = 2  # Number of equations
    T = 1.0  # Final time
    dt = 0.1  # Time step
    problem_name = "My New Problem"
    
    # 2. Physical parameters
    parameters = np.array([param1, param2, param3, param4])
    
    # 3. Define functions (chemotaxis, sources, solutions, etc.)
    def chi(x): return np.ones_like(x)
    def dchi(x): return np.zeros_like(x)
    def source_u(s, t): return 0.0 * s
    def source_phi(s, t): return 0.0 * s
    def initial_u(s, t=0.0): return np.ones_like(s)
    def initial_phi(s, t=0.0): return np.zeros_like(s)
    
    # 4. Create geometry
    geometry = DomainGeometry("my_geometry")
    # Add domains using geometry.add_domain(...)
    
    # 5. Create problems from geometry
    problems = []
    discretizations = []
    for domain_id in range(geometry.num_domains()):
        domain_info = geometry.get_domain(domain_id)
        # Create Problem and Discretization objects
    
    # 6. Set up constraints
    constraint_manager = ConstraintManager()
    # Add boundary and interface constraints
    
    # 7. Return framework components
    return problems, global_discretization, constraint_manager, problem_name
\end{lstlisting}

\subsection{Keller-Segel Problems}

For chemotaxis problems, include:

\begin{lstlisting}[language=Python, caption={Keller-Segel Problem Setup}]
# Chemotaxis sensitivity function
def chi(x):
    k1, k2 = 3.9e-9, 5.e-6
    return k1 / (k2 + x)**2

def dchi(x):
    k1, k2 = 3.9e-9, 5.e-6
    return -k1 * 2 / (k2 + x)**3

# Set chemotaxis for all problems
for problem in problems:
    problem.set_chemotaxis(chi, dchi)
    problem.set_force(0, source_u)      # Cell equation source
    problem.set_force(1, source_phi)    # Chemical equation source
\end{lstlisting}

\subsection{Organ-on-Chip Problems}

For microfluidic systems, focus on:

\begin{lstlisting}[language=Python, caption={Organ-on-Chip Problem Setup}]
# Multi-compartment setup
compartments = ["inlet", "cell_chamber", "outlet", "waste"]

# Different parameters per compartment
parameters_list = [
    np.array([D1, v1, k1, 0.0]),     # Inlet: high flow
    np.array([D2, v2, k2, k_cell]),  # Cell chamber: cell interaction
    np.array([D3, v3, k3, 0.0]),     # Outlet: medium flow
    np.array([D4, v4, k4, 0.0])      # Waste: low flow
]

# Junction conditions with permeabilities
permeabilities = [0.8, 1.0, 0.9]  # Between compartments
\end{lstlisting}

\section{Geometry Module Guide}

\subsection{Simple Linear Network}

\begin{lstlisting}[language=Python, caption={Linear Network Geometry}]
geometry = DomainGeometry("linear_chain")

# Add sequential domains
geometry.add_domain(
    extrema_start=(0.0, 0.0),
    extrema_end=(1.0, 0.0),
    name="segment1"
)

geometry.add_domain(
    extrema_start=(1.0, 0.0),
    extrema_end=(2.0, 0.0),
    name="segment2"
)
\end{lstlisting}

\subsection{T-Junction Network}

\begin{lstlisting}[language=Python, caption={T-Junction Geometry}]
geometry = DomainGeometry("t_junction")

# Main channel
geometry.add_domain(
    extrema_start=(0.0, -1.0),
    extrema_end=(0.0, 1.0),
    name="main_channel"
)

# Side branch
geometry.add_domain(
    extrema_start=(0.0, 0.0),
    extrema_end=(1.0, 0.0),
    name="side_branch"
)
\end{lstlisting}

\subsection{Grid Network}

\begin{lstlisting}[language=Python, caption={Grid Network Geometry}]
geometry = DomainGeometry("grid_network")

# Vertical segments
for i, x_pos in enumerate([-0.5, 0.5]):
    geometry.add_domain(
        extrema_start=(x_pos, 0.0),
        extrema_end=(x_pos, 1.0),
        name=f"vertical_{i}"
    )

# Horizontal connectors
for i, y_pos in enumerate([0.2, 0.4, 0.6, 0.8]):
    geometry.add_domain(
        extrema_start=(-0.5, y_pos),
        extrema_end=(0.5, y_pos),
        name=f"horizontal_{i}"
    )
\end{lstlisting}

\subsection{Complex Branching Network}

\begin{lstlisting}[language=Python, caption={Branching Network Geometry}]
geometry = DomainGeometry("branching_network")

# Main trunk
geometry.add_domain(
    extrema_start=(0.0, 0.0),
    extrema_end=(0.0, 2.0),
    name="trunk"
)

# Branches at different levels
branch_angles = [30, 60, 120, 150]  # degrees
for i, angle in enumerate(branch_angles):
    angle_rad = np.radians(angle)
    length = 1.0
    end_x = length * np.cos(angle_rad)
    end_y = 1.0 + length * np.sin(angle_rad)
    
    geometry.add_domain(
        extrema_start=(0.0, 1.0),
        extrema_end=(end_x, end_y),
        name=f"branch_{i}"
    )
\end{lstlisting}

\section{Visualization System}

\subsection{2D Curve Plots}

Best for analyzing solution profiles along individual domains:

\begin{lstlisting}[language=Python, caption={2D Curve Plotting}]
plotter.plot_2d_curves(
    trace_solutions=solutions,
    title="Solution Profiles",
    show_mesh_points=True,
    save_filename="solution_curves.png"
)
\end{lstlisting}

\textbf{Features:}
\begin{itemize}
    \item Separate subplot per domain
    \item All equations shown in each domain
    \item Mesh point markers
    \item Domain boundary indicators
\end{itemize}

\subsection{Flat 3D View}

Ideal for understanding network topology with solution values:

\begin{lstlisting}[language=Python, caption={Flat 3D Visualization}]
plotter.plot_flat_3d(
    trace_solutions=solutions,
    equation_idx=0,
    view_angle=(30, 45),
    save_filename="network_3d.png"
)
\end{lstlisting}

\textbf{Features:}
\begin{itemize}
    \item Network segments in xy-plane
    \item Solution values as colored scatter points above
    \item Connecting lines from segments to solution points
    \item Rotatable 3D view
\end{itemize}

\subsection{Bird's Eye View}

Perfect for network-level solution analysis:

\begin{lstlisting}[language=Python, caption={Bird's Eye View Plotting}]
plotter.plot_birdview(
    trace_solutions=solutions,
    equation_idx=0,
    time=current_time,
    save_filename="network_overview.png"
)
\end{lstlisting}

\textbf{Features:}
\begin{itemize}
    \item Top-down network view
    \item Color-coded segment thickness
    \item Solution point markers
    \item Clean network overview
\end{itemize}

\clearpage
\thispagestyle{empty} % Remove headers/footers
\vspace*{\fill}

\begin{center}
	{\Huge \textbf{Chapter Title}}\\[1cm]
	{\Large Subtitle or Description (if any)}\\[2cm]
%	\includegraphics[width=0.3\textwidth]{path/to/your/image.png} % Optional image
\end{center}

\vspace*{\fill}
\clearpage

\input{domain_geometry_api}
\input{problem_module_api_accurate}
\input{constraints_module_api}
% Discretization Module API Documentation
% To be included in master LaTeX document
%
% Usage: % Discretization Module API Documentation
% To be included in master LaTeX document
%
% Usage: % Discretization Module API Documentation
% To be included in master LaTeX document
%
% Usage: \input{docs/discretization_module_api}

\section{Discretization Module API Reference}
\label{sec:discretization_module_api}

This section provides a comprehensive reference for the Discretization classes (\texttt{bionetflux.core.discretization}) based on the actual implementation. The module contains classes for spatial and temporal discretization management using finite elements.

\subsection{Module Overview}

The discretization module contains two main classes:
\begin{itemize}
    \item \texttt{Discretization}: Single-domain spatial discretization using finite elements
    \item \texttt{GlobalDiscretization}: Multi-domain coordinator with time stepping parameters
\end{itemize}

\subsection{Module Dependencies}

\begin{lstlisting}[language=Python, caption=Module Dependencies]
import numpy as np
from typing import List, Optional
\end{lstlisting}

\subsection{Discretization Class}
\label{subsec:discretization_class}

The main class for single-domain spatial discretization using finite elements. Handles spatial mesh generation and element properties.

\subsubsection{Constructor}

\paragraph{\_\_init\_\_()}\leavevmode
\begin{lstlisting}[language=Python, caption=Discretization Constructor]
def __init__(self, n_elements: int, domain_start: float = 0.0, 
             domain_length: float = 1.0, stab_constant: float = 1.0)
\end{lstlisting}

\textbf{Parameters:}
\begin{itemize}
    \item \texttt{n\_elements}: Number of elements in the mesh
    \item \texttt{domain\_start}: Domain start coordinate (default: 0.0)
    \item \texttt{domain\_length}: Domain length (default: 1.0)
    \item \texttt{stab\_constant}: Stabilization constant (default: 1.0)
\end{itemize}

\textbf{Description:} Creates a spatial discretization and automatically generates the mesh by calling \texttt{\_generate\_mesh()}.

\textbf{Usage Examples:}
\begin{lstlisting}[language=Python, caption=Discretization Constructor Usage]
# Basic discretization with 20 elements
disc1 = Discretization(n_elements=20)

# Custom domain discretization
disc2 = Discretization(
    n_elements=40,
    domain_start=0.0,  
    domain_length=1.0,
    stab_constant=1.0
)

# Fine mesh discretization on different domain
disc3 = Discretization(
    n_elements=100,
    domain_start=-1.0,
    domain_length=2.0,
    stab_constant=0.5
)
\end{lstlisting}

\subsubsection{Core Attributes}

\begin{longtable}{|p{3.2cm}|p{3cm}|p{7cm}|}
\hline
\textbf{Attribute} & \textbf{Type} & \textbf{Description} \\
\hline
\endhead

\texttt{n\_elements} & \texttt{int} & Number of elements in the mesh \\
\hline

\texttt{domain\_start} & \texttt{float} & Start coordinate of the domain \\
\hline

\texttt{domain\_length} & \texttt{float} & Length of the domain \\
\hline

\texttt{stab\_constant} & \texttt{float} & Stabilization constant \\
\hline

\texttt{n\_nodes} & \texttt{int} & Number of nodes: \texttt{n\_elements + 1} \\
\hline

\texttt{element\_length} & \texttt{float} & Uniform element size: \texttt{domain\_length / n\_elements} \\
\hline

\texttt{nodes} & \texttt{np.ndarray} & Node coordinates array (generated by \texttt{\_generate\_mesh}) \\
\hline

\texttt{elements} & \texttt{np.ndarray} & Element connectivity array: \texttt{[[i, i+1] for i in range(n\_elements)]} \\
\hline

\texttt{element\_centers} & \texttt{np.ndarray} & Element center coordinates \\
\hline

\texttt{tau} & \texttt{np.ndarray} & Stabilization parameters per equation (set by \texttt{set\_tau}) \\
\hline

\end{longtable}

\subsubsection{Private Methods}

\paragraph{\_generate\_mesh()}\leavevmode
\begin{lstlisting}[language=Python, caption=Generate Mesh Method]
def _generate_mesh(self)
\end{lstlisting}

\textbf{Description:} Private method that generates the spatial mesh nodes and connectivity. Called automatically during initialization.

\textbf{Implementation:}
\begin{itemize}
    \item Creates \texttt{self.nodes} using \texttt{np.linspace(domain\_start, domain\_start + domain\_length, n\_nodes)}
    \item Creates \texttt{self.elements} as \texttt{np.array([[i, i+1] for i in range(n\_elements)])}
    \item Creates \texttt{self.element\_centers} as \texttt{nodes[:-1] + element\_length / 2}
\end{itemize}

\textbf{Usage:} This method is called automatically during object initialization and should not be called directly.

\subsubsection{Public Methods}

\paragraph{get\_mesh\_info()}\leavevmode
\begin{lstlisting}[language=Python, caption=Get Mesh Info Method]
def get_mesh_info(self) -> dict
\end{lstlisting}

\textbf{Returns:} \texttt{dict} - Dictionary containing complete mesh information

\textbf{Dictionary Keys:}
\begin{itemize}
    \item \texttt{n\_elements}: Number of elements
    \item \texttt{n\_nodes}: Number of nodes
    \item \texttt{element\_length}: Element length
    \item \texttt{nodes}: Node coordinates array
    \item \texttt{elements}: Element connectivity array
    \item \texttt{element\_centers}: Element center coordinates
    \item \texttt{domain\_start}: Domain start coordinate
    \item \texttt{domain\_length}: Domain length
    \item \texttt{stab\_constant}: Stabilization constant
\end{itemize}

\textbf{Usage:}
\begin{lstlisting}[language=Python, caption=Get Mesh Info Usage]
disc = Discretization(n_elements=10, domain_start=0.0, domain_length=2.0)
mesh_info = disc.get_mesh_info()
print(f"Elements: {mesh_info['n_elements']}")
print(f"Nodes: {mesh_info['n_nodes']}")
print(f"Element length: {mesh_info['element_length']}")
print(f"First few nodes: {mesh_info['nodes'][:5]}")
\end{lstlisting}

\paragraph{set\_tau()}\leavevmode
\begin{lstlisting}[language=Python, caption=Set Tau Method]
def set_tau(self, tau_values: List[float])
\end{lstlisting}

\textbf{Parameters:}
\begin{itemize}
    \item \texttt{tau\_values}: List of stabilization parameters for each equation
\end{itemize}

\textbf{Side Effects:} Sets \texttt{self.tau} as a numpy array

\textbf{Validation:} Raises \texttt{ValueError} if \texttt{tau\_values} list is empty

\textbf{Usage:}
\begin{lstlisting}[language=Python, caption=Set Tau Usage]
# Keller-Segel problem (2 equations)
ks_disc = Discretization(n_elements=20)
ks_disc.set_tau([1.0, 1.0])  # [tau_u, tau_phi]

# OrganOnChip problem (4 equations)
ooc_disc = Discretization(n_elements=40)
ooc_disc.set_tau([1.0, 1.0, 1.0, 1.0])  # [tu, to, tv, tp]

# Different stabilization parameters
custom_disc = Discretization(n_elements=30)
custom_disc.set_tau([0.5, 2.0, 1.5])  # Custom values per equation

# This will raise ValueError
try:
    disc.set_tau([])  # Empty list
except ValueError as e:
    print(f"Error: {e}")
\end{lstlisting}



\subsection{GlobalDiscretization Class}
\label{subsec:globaldiscretization_class}

Global discretization class managing multiple spatial domains and time discretization. Coordinates temporal evolution across all domains.

\subsubsection{Constructor}

\paragraph{\_\_init\_\_()}\leavevmode
\begin{lstlisting}[language=Python, caption=GlobalDiscretization Constructor]
def __init__(self, spatial_discretizations: List[Discretization])
\end{lstlisting}

\textbf{Parameters:}
\begin{itemize}
    \item \texttt{spatial\_discretizations}: List of Discretization instances for each domain
\end{itemize}

\textbf{Description:} Creates a global discretization coordinator and automatically computes global mesh information by calling \texttt{\_compute\_global\_info()}.

\textbf{Usage:}
\begin{lstlisting}[language=Python, caption=GlobalDiscretization Usage]
# Single domain
disc1 = Discretization(n_elements=20, domain_start=0.0, domain_length=1.0)
global_disc = GlobalDiscretization([disc1])

# Multi-domain network
main_disc = Discretization(n_elements=30, domain_start=0.0, domain_length=1.0)
branch1_disc = Discretization(n_elements=20, domain_start=1.0, domain_length=0.8)
branch2_disc = Discretization(n_elements=20, domain_start=1.0, domain_length=0.8)

multi_global_disc = GlobalDiscretization([main_disc, branch1_disc, branch2_disc])
\end{lstlisting}

\subsubsection{Core Attributes}

\begin{longtable}{|p{4.6cm}|p{4cm}|p{5.5cm}|}
\hline
\textbf{Attribute} & \textbf{Type} & \textbf{Description} \\
\hline
\endhead

\texttt{spatial\_discretizations} & \texttt{List[Discretization]} & List of spatial discretizations for each domain \\
\hline

\texttt{n\_domains} & \texttt{int} & Number of domains: \texttt{len(spatial\_discretizations)} \\
\hline

\texttt{dt} & \texttt{Optional[float]} & Global time step size (initially None) \\
\hline

\texttt{T} & \texttt{Optional[float]} & Final time (initially None) \\
\hline

\texttt{n\_time\_steps} & \texttt{Optional[int]} & Number of time steps: \texttt{int(np.ceil(T/dt))} \\
\hline

\texttt{time\_points} & \texttt{Optional[np.ndarray]} & Array of time points from 0 to T \\
\hline

\texttt{total\_elements} & \texttt{int} & Sum of elements across all domains \\
\hline

\texttt{total\_nodes} & \texttt{int} & Sum of nodes across all domains \\
\hline

\end{longtable}

\subsubsection{Private Methods}

\paragraph{\_compute\_global\_info()}\leavevmode
\begin{lstlisting}[language=Python, caption=Compute Global Info Method]
def _compute_global_info(self)
\end{lstlisting}

\textbf{Description:} Private method that computes global mesh information from all domains. Called automatically during initialization.

\textbf{Implementation:}
\begin{itemize}
    \item Sets \texttt{self.total\_elements} as sum of elements from all spatial discretizations
    \item Sets \texttt{self.total\_nodes} as sum of nodes from all spatial discretizations
\end{itemize}

\textbf{Usage:} This method is called automatically during object initialization and should not be called directly.

\subsubsection{Public Methods}

\paragraph{set\_time\_parameters()}\leavevmode
\begin{lstlisting}[language=Python, caption=Set Time Parameters Method]
def set_time_parameters(self, dt: float, T: float)
\end{lstlisting}

\textbf{Parameters:}
\begin{itemize}
    \item \texttt{dt}: Time step size
    \item \texttt{T}: Final time
\end{itemize}

\textbf{Side Effects:} 
\begin{itemize}
    \item Sets \texttt{self.dt}, \texttt{self.T}, \texttt{self.n\_time\_steps}, and \texttt{self.time\_points}
    \item Calculates \texttt{n\_time\_steps = int(np.ceil(T / dt))}
    \item Creates \texttt{time\_points = np.linspace(0, T, n\_time\_steps + 1)}
\end{itemize}

\textbf{Usage:}
\begin{lstlisting}[language=Python, caption=Set Time Parameters Usage]
global_disc = GlobalDiscretization([disc1, disc2])
global_disc.set_time_parameters(dt=0.01, T=1.0)

print(f"Time step: {global_disc.dt}")
print(f"Final time: {global_disc.T}")
print(f"Number of time steps: {global_disc.n_time_steps}")
print(f"First few time points: {global_disc.time_points[:5]}")
\end{lstlisting}

\paragraph{get\_spatial\_discretization()}\leavevmode
\begin{lstlisting}[language=Python, caption=Get Spatial Discretization Method]
def get_spatial_discretization(self, domain_index: int) -> Discretization
\end{lstlisting}

\textbf{Parameters:}
\begin{itemize}
    \item \texttt{domain\_index}: Domain index (0 to \texttt{n\_domains-1})
\end{itemize}

\textbf{Returns:} \texttt{Discretization} - Spatial discretization for specified domain

\textbf{Raises:} \texttt{IndexError} if domain\_index is out of range

\textbf{Usage:}
\begin{lstlisting}[language=Python, caption=Get Spatial Discretization Usage]
try:
    domain_0_disc = global_disc.get_spatial_discretization(0)
    print(f"Domain 0 elements: {domain_0_disc.n_elements}")
    print(f"Domain 0 nodes: {domain_0_disc.n_nodes}")
    
    # This will raise IndexError if only 2 domains exist
    domain_5_disc = global_disc.get_spatial_discretization(5)
except IndexError as e:
    print(f"Error: {e}")
\end{lstlisting}

\paragraph{get\_time\_info()}\leavevmode
\begin{lstlisting}[language=Python, caption=Get Time Info Method]
def get_time_info(self) -> dict
\end{lstlisting}

\textbf{Returns:} \texttt{dict} - Dictionary containing time discretization information

\textbf{Dictionary Keys:}
\begin{itemize}
    \item \texttt{dt}: Time step size
    \item \texttt{T}: Final time
    \item \texttt{n\_time\_steps}: Number of time steps
    \item \texttt{time\_points}: Array of time points
\end{itemize}

\textbf{Usage:}
\begin{lstlisting}[language=Python, caption=Get Time Info Usage]
global_disc.set_time_parameters(dt=0.01, T=1.0)
time_info = global_disc.get_time_info()
print(f"Time step: {time_info['dt']}")
print(f"Final time: {time_info['T']}")
print(f"Number of time steps: {time_info['n_time_steps']}")
print(f"Time points shape: {time_info['time_points'].shape}")
\end{lstlisting}

\paragraph{get\_global\_info()}\leavevmode
\begin{lstlisting}[language=Python, caption=Get Global Info Method]
def get_global_info(self) -> dict
\end{lstlisting}

\textbf{Returns:} \texttt{dict} - Dictionary containing complete global discretization information

\textbf{Dictionary Keys:}
\begin{itemize}
    \item \texttt{n\_domains}: Number of domains
    \item \texttt{total\_elements}: Total number of elements across all domains
    \item \texttt{total\_nodes}: Total number of nodes across all domains
    \item \texttt{time\_info}: Time discretization information (from \texttt{get\_time\_info()})
    \item \texttt{spatial\_discretizations}: List of mesh information for each domain
\end{itemize}

\textbf{Usage:}
\begin{lstlisting}[language=Python, caption=Get Global Info Usage]
global_disc.set_time_parameters(dt=0.01, T=1.0)
global_info = global_disc.get_global_info()

print(f"Number of domains: {global_info['n_domains']}")
print(f"Total elements: {global_info['total_elements']}")
print(f"Total nodes: {global_info['total_nodes']}")

# Access time information
time_info = global_info['time_info']
print(f"Time step: {time_info['dt']}")

# Access spatial discretization info for each domain
for i, spatial_info in enumerate(global_info['spatial_discretizations']):
    print(f"Domain {i}: {spatial_info['n_elements']} elements")
\end{lstlisting}

\subsection{Complete Usage Examples}
\label{subsec:discretization_complete_examples}

\subsubsection{Single Domain Setup}

\begin{lstlisting}[language=Python, caption=Single Domain Discretization Setup]
from bionetflux.core.discretization import Discretization, GlobalDiscretization

# Create spatial discretization for a single domain
spatial_disc = Discretization(
    n_elements=40,
    domain_start=0.0,
    domain_length=1.0,
    stab_constant=1.0
)

# Set stabilization parameters (e.g., for Keller-Segel: 2 equations)
spatial_disc.set_tau([1.0, 1.0])

# Create global discretization
global_disc = GlobalDiscretization([spatial_disc])

# Set time parameters
global_disc.set_time_parameters(dt=0.01, T=1.0)

# Access mesh information
mesh_info = spatial_disc.get_mesh_info()
print(f"Elements: {mesh_info['n_elements']}")
print(f"Nodes: {mesh_info['n_nodes']}")
print(f"Element length: {mesh_info['element_length']}")

# Access global information
global_info = global_disc.get_global_info()
print(f"Total elements: {global_info['total_elements']}")
print(f"Time steps: {global_info['time_info']['n_time_steps']}")
\end{lstlisting}

\subsubsection{Multi-Domain Network Setup}

\begin{lstlisting}[language=Python, caption=Multi-Domain Network Discretization]
# Create multiple spatial discretizations for network domains
main_disc = Discretization(n_elements=30, domain_start=0.0, domain_length=1.0)
branch1_disc = Discretization(n_elements=20, domain_start=1.0, domain_length=0.8)
branch2_disc = Discretization(n_elements=20, domain_start=1.0, domain_length=0.8)

# Set stabilization parameters for each domain
main_disc.set_tau([1.0, 1.0, 1.0, 1.0])     # 4-equation system
branch1_disc.set_tau([1.0, 1.0, 1.0, 1.0])  # 4-equation system  
branch2_disc.set_tau([1.0, 1.0, 1.0, 1.0])  # 4-equation system

# Create global discretization for the network
network_global_disc = GlobalDiscretization([main_disc, branch1_disc, branch2_disc])

# Set global time parameters
network_global_disc.set_time_parameters(dt=0.01, T=0.5)

# Access individual domain discretizations
domain_0 = network_global_disc.get_spatial_discretization(0)
print(f"Main domain elements: {domain_0.n_elements}")

# Get global statistics
global_info = network_global_disc.get_global_info()
print(f"Network has {global_info['n_domains']} domains")
print(f"Total elements: {global_info['total_elements']}")
print(f"Total nodes: {global_info['total_nodes']}")
\end{lstlisting}

\subsection{Method Summary Tables}
\label{subsec:discretization_method_summary}

\subsubsection{Discretization Class Methods}

\begin{longtable}{|p{4.0cm}|p{2.5cm}|p{6.5cm}|}
\hline
\textbf{Method} & \textbf{Returns} & \textbf{Purpose} \\
\hline
\endhead

\texttt{\_generate\_mesh} & \texttt{None} & Generate spatial mesh (private, called during init) \\
\hline

\texttt{get\_mesh\_info} & \texttt{dict} & Return complete mesh information dictionary \\
\hline

\texttt{set\_tau} & \texttt{None} & Set stabilization parameters per equation \\
\hline

\end{longtable}

\subsubsection{GlobalDiscretization Class Methods}

\begin{longtable}{|p{5.2cm}|p{2.7cm}|p{6cm}|}
\hline
\textbf{Method} & \textbf{Returns} & \textbf{Purpose} \\
\hline
\endhead

\texttt{\_compute\_global\_info} & \texttt{None} & Compute global mesh info (private, called during init) \\
\hline

\texttt{set\_time\_parameters} & \texttt{None} & Set global time step and final time \\
\hline

\texttt{get\_spatial\_discretization} & \texttt{Discretization} & Access specific domain discretization \\
\hline

\texttt{get\_time\_info} & \texttt{dict} & Return time discretization information \\
\hline

\texttt{get\_global\_info} & \texttt{dict} & Return complete global discretization information \\
\hline

\end{longtable}

This documentation provides an exact reference for the Discretization module based on the actual BioNetFlux implementation in \texttt{bionetflux.core.discretization}.

\begin{todobox}
	\begin{itemize}
		\item Check doubling of information with geometry module
		\item Should the stabilization parameter be an attribute of the Discretization class? Probably not. Evaluate options. Modification has (positive) impact for {\tt static\_condensation}
		\item Allow for non uniform grids
		\item Allow for non uniform time stepping
		\item Include P\'eclet number checks and checks of requirements on {\tt dt} vs {\tt h}
		\end{itemize}
	\end{todobox}

% End of discretization module API documentation


\section{Discretization Module API Reference}
\label{sec:discretization_module_api}

This section provides a comprehensive reference for the Discretization classes (\texttt{bionetflux.core.discretization}) based on the actual implementation. The module contains classes for spatial and temporal discretization management using finite elements.

\subsection{Module Overview}

The discretization module contains two main classes:
\begin{itemize}
    \item \texttt{Discretization}: Single-domain spatial discretization using finite elements
    \item \texttt{GlobalDiscretization}: Multi-domain coordinator with time stepping parameters
\end{itemize}

\subsection{Module Dependencies}

\begin{lstlisting}[language=Python, caption=Module Dependencies]
import numpy as np
from typing import List, Optional
\end{lstlisting}

\subsection{Discretization Class}
\label{subsec:discretization_class}

The main class for single-domain spatial discretization using finite elements. Handles spatial mesh generation and element properties.

\subsubsection{Constructor}

\paragraph{\_\_init\_\_()}\leavevmode
\begin{lstlisting}[language=Python, caption=Discretization Constructor]
def __init__(self, n_elements: int, domain_start: float = 0.0, 
             domain_length: float = 1.0, stab_constant: float = 1.0)
\end{lstlisting}

\textbf{Parameters:}
\begin{itemize}
    \item \texttt{n\_elements}: Number of elements in the mesh
    \item \texttt{domain\_start}: Domain start coordinate (default: 0.0)
    \item \texttt{domain\_length}: Domain length (default: 1.0)
    \item \texttt{stab\_constant}: Stabilization constant (default: 1.0)
\end{itemize}

\textbf{Description:} Creates a spatial discretization and automatically generates the mesh by calling \texttt{\_generate\_mesh()}.

\textbf{Usage Examples:}
\begin{lstlisting}[language=Python, caption=Discretization Constructor Usage]
# Basic discretization with 20 elements
disc1 = Discretization(n_elements=20)

# Custom domain discretization
disc2 = Discretization(
    n_elements=40,
    domain_start=0.0,  
    domain_length=1.0,
    stab_constant=1.0
)

# Fine mesh discretization on different domain
disc3 = Discretization(
    n_elements=100,
    domain_start=-1.0,
    domain_length=2.0,
    stab_constant=0.5
)
\end{lstlisting}

\subsubsection{Core Attributes}

\begin{longtable}{|p{3.2cm}|p{3cm}|p{7cm}|}
\hline
\textbf{Attribute} & \textbf{Type} & \textbf{Description} \\
\hline
\endhead

\texttt{n\_elements} & \texttt{int} & Number of elements in the mesh \\
\hline

\texttt{domain\_start} & \texttt{float} & Start coordinate of the domain \\
\hline

\texttt{domain\_length} & \texttt{float} & Length of the domain \\
\hline

\texttt{stab\_constant} & \texttt{float} & Stabilization constant \\
\hline

\texttt{n\_nodes} & \texttt{int} & Number of nodes: \texttt{n\_elements + 1} \\
\hline

\texttt{element\_length} & \texttt{float} & Uniform element size: \texttt{domain\_length / n\_elements} \\
\hline

\texttt{nodes} & \texttt{np.ndarray} & Node coordinates array (generated by \texttt{\_generate\_mesh}) \\
\hline

\texttt{elements} & \texttt{np.ndarray} & Element connectivity array: \texttt{[[i, i+1] for i in range(n\_elements)]} \\
\hline

\texttt{element\_centers} & \texttt{np.ndarray} & Element center coordinates \\
\hline

\texttt{tau} & \texttt{np.ndarray} & Stabilization parameters per equation (set by \texttt{set\_tau}) \\
\hline

\end{longtable}

\subsubsection{Private Methods}

\paragraph{\_generate\_mesh()}\leavevmode
\begin{lstlisting}[language=Python, caption=Generate Mesh Method]
def _generate_mesh(self)
\end{lstlisting}

\textbf{Description:} Private method that generates the spatial mesh nodes and connectivity. Called automatically during initialization.

\textbf{Implementation:}
\begin{itemize}
    \item Creates \texttt{self.nodes} using \texttt{np.linspace(domain\_start, domain\_start + domain\_length, n\_nodes)}
    \item Creates \texttt{self.elements} as \texttt{np.array([[i, i+1] for i in range(n\_elements)])}
    \item Creates \texttt{self.element\_centers} as \texttt{nodes[:-1] + element\_length / 2}
\end{itemize}

\textbf{Usage:} This method is called automatically during object initialization and should not be called directly.

\subsubsection{Public Methods}

\paragraph{get\_mesh\_info()}\leavevmode
\begin{lstlisting}[language=Python, caption=Get Mesh Info Method]
def get_mesh_info(self) -> dict
\end{lstlisting}

\textbf{Returns:} \texttt{dict} - Dictionary containing complete mesh information

\textbf{Dictionary Keys:}
\begin{itemize}
    \item \texttt{n\_elements}: Number of elements
    \item \texttt{n\_nodes}: Number of nodes
    \item \texttt{element\_length}: Element length
    \item \texttt{nodes}: Node coordinates array
    \item \texttt{elements}: Element connectivity array
    \item \texttt{element\_centers}: Element center coordinates
    \item \texttt{domain\_start}: Domain start coordinate
    \item \texttt{domain\_length}: Domain length
    \item \texttt{stab\_constant}: Stabilization constant
\end{itemize}

\textbf{Usage:}
\begin{lstlisting}[language=Python, caption=Get Mesh Info Usage]
disc = Discretization(n_elements=10, domain_start=0.0, domain_length=2.0)
mesh_info = disc.get_mesh_info()
print(f"Elements: {mesh_info['n_elements']}")
print(f"Nodes: {mesh_info['n_nodes']}")
print(f"Element length: {mesh_info['element_length']}")
print(f"First few nodes: {mesh_info['nodes'][:5]}")
\end{lstlisting}

\paragraph{set\_tau()}\leavevmode
\begin{lstlisting}[language=Python, caption=Set Tau Method]
def set_tau(self, tau_values: List[float])
\end{lstlisting}

\textbf{Parameters:}
\begin{itemize}
    \item \texttt{tau\_values}: List of stabilization parameters for each equation
\end{itemize}

\textbf{Side Effects:} Sets \texttt{self.tau} as a numpy array

\textbf{Validation:} Raises \texttt{ValueError} if \texttt{tau\_values} list is empty

\textbf{Usage:}
\begin{lstlisting}[language=Python, caption=Set Tau Usage]
# Keller-Segel problem (2 equations)
ks_disc = Discretization(n_elements=20)
ks_disc.set_tau([1.0, 1.0])  # [tau_u, tau_phi]

# OrganOnChip problem (4 equations)
ooc_disc = Discretization(n_elements=40)
ooc_disc.set_tau([1.0, 1.0, 1.0, 1.0])  # [tu, to, tv, tp]

# Different stabilization parameters
custom_disc = Discretization(n_elements=30)
custom_disc.set_tau([0.5, 2.0, 1.5])  # Custom values per equation

# This will raise ValueError
try:
    disc.set_tau([])  # Empty list
except ValueError as e:
    print(f"Error: {e}")
\end{lstlisting}



\subsection{GlobalDiscretization Class}
\label{subsec:globaldiscretization_class}

Global discretization class managing multiple spatial domains and time discretization. Coordinates temporal evolution across all domains.

\subsubsection{Constructor}

\paragraph{\_\_init\_\_()}\leavevmode
\begin{lstlisting}[language=Python, caption=GlobalDiscretization Constructor]
def __init__(self, spatial_discretizations: List[Discretization])
\end{lstlisting}

\textbf{Parameters:}
\begin{itemize}
    \item \texttt{spatial\_discretizations}: List of Discretization instances for each domain
\end{itemize}

\textbf{Description:} Creates a global discretization coordinator and automatically computes global mesh information by calling \texttt{\_compute\_global\_info()}.

\textbf{Usage:}
\begin{lstlisting}[language=Python, caption=GlobalDiscretization Usage]
# Single domain
disc1 = Discretization(n_elements=20, domain_start=0.0, domain_length=1.0)
global_disc = GlobalDiscretization([disc1])

# Multi-domain network
main_disc = Discretization(n_elements=30, domain_start=0.0, domain_length=1.0)
branch1_disc = Discretization(n_elements=20, domain_start=1.0, domain_length=0.8)
branch2_disc = Discretization(n_elements=20, domain_start=1.0, domain_length=0.8)

multi_global_disc = GlobalDiscretization([main_disc, branch1_disc, branch2_disc])
\end{lstlisting}

\subsubsection{Core Attributes}

\begin{longtable}{|p{4.6cm}|p{4cm}|p{5.5cm}|}
\hline
\textbf{Attribute} & \textbf{Type} & \textbf{Description} \\
\hline
\endhead

\texttt{spatial\_discretizations} & \texttt{List[Discretization]} & List of spatial discretizations for each domain \\
\hline

\texttt{n\_domains} & \texttt{int} & Number of domains: \texttt{len(spatial\_discretizations)} \\
\hline

\texttt{dt} & \texttt{Optional[float]} & Global time step size (initially None) \\
\hline

\texttt{T} & \texttt{Optional[float]} & Final time (initially None) \\
\hline

\texttt{n\_time\_steps} & \texttt{Optional[int]} & Number of time steps: \texttt{int(np.ceil(T/dt))} \\
\hline

\texttt{time\_points} & \texttt{Optional[np.ndarray]} & Array of time points from 0 to T \\
\hline

\texttt{total\_elements} & \texttt{int} & Sum of elements across all domains \\
\hline

\texttt{total\_nodes} & \texttt{int} & Sum of nodes across all domains \\
\hline

\end{longtable}

\subsubsection{Private Methods}

\paragraph{\_compute\_global\_info()}\leavevmode
\begin{lstlisting}[language=Python, caption=Compute Global Info Method]
def _compute_global_info(self)
\end{lstlisting}

\textbf{Description:} Private method that computes global mesh information from all domains. Called automatically during initialization.

\textbf{Implementation:}
\begin{itemize}
    \item Sets \texttt{self.total\_elements} as sum of elements from all spatial discretizations
    \item Sets \texttt{self.total\_nodes} as sum of nodes from all spatial discretizations
\end{itemize}

\textbf{Usage:} This method is called automatically during object initialization and should not be called directly.

\subsubsection{Public Methods}

\paragraph{set\_time\_parameters()}\leavevmode
\begin{lstlisting}[language=Python, caption=Set Time Parameters Method]
def set_time_parameters(self, dt: float, T: float)
\end{lstlisting}

\textbf{Parameters:}
\begin{itemize}
    \item \texttt{dt}: Time step size
    \item \texttt{T}: Final time
\end{itemize}

\textbf{Side Effects:} 
\begin{itemize}
    \item Sets \texttt{self.dt}, \texttt{self.T}, \texttt{self.n\_time\_steps}, and \texttt{self.time\_points}
    \item Calculates \texttt{n\_time\_steps = int(np.ceil(T / dt))}
    \item Creates \texttt{time\_points = np.linspace(0, T, n\_time\_steps + 1)}
\end{itemize}

\textbf{Usage:}
\begin{lstlisting}[language=Python, caption=Set Time Parameters Usage]
global_disc = GlobalDiscretization([disc1, disc2])
global_disc.set_time_parameters(dt=0.01, T=1.0)

print(f"Time step: {global_disc.dt}")
print(f"Final time: {global_disc.T}")
print(f"Number of time steps: {global_disc.n_time_steps}")
print(f"First few time points: {global_disc.time_points[:5]}")
\end{lstlisting}

\paragraph{get\_spatial\_discretization()}\leavevmode
\begin{lstlisting}[language=Python, caption=Get Spatial Discretization Method]
def get_spatial_discretization(self, domain_index: int) -> Discretization
\end{lstlisting}

\textbf{Parameters:}
\begin{itemize}
    \item \texttt{domain\_index}: Domain index (0 to \texttt{n\_domains-1})
\end{itemize}

\textbf{Returns:} \texttt{Discretization} - Spatial discretization for specified domain

\textbf{Raises:} \texttt{IndexError} if domain\_index is out of range

\textbf{Usage:}
\begin{lstlisting}[language=Python, caption=Get Spatial Discretization Usage]
try:
    domain_0_disc = global_disc.get_spatial_discretization(0)
    print(f"Domain 0 elements: {domain_0_disc.n_elements}")
    print(f"Domain 0 nodes: {domain_0_disc.n_nodes}")
    
    # This will raise IndexError if only 2 domains exist
    domain_5_disc = global_disc.get_spatial_discretization(5)
except IndexError as e:
    print(f"Error: {e}")
\end{lstlisting}

\paragraph{get\_time\_info()}\leavevmode
\begin{lstlisting}[language=Python, caption=Get Time Info Method]
def get_time_info(self) -> dict
\end{lstlisting}

\textbf{Returns:} \texttt{dict} - Dictionary containing time discretization information

\textbf{Dictionary Keys:}
\begin{itemize}
    \item \texttt{dt}: Time step size
    \item \texttt{T}: Final time
    \item \texttt{n\_time\_steps}: Number of time steps
    \item \texttt{time\_points}: Array of time points
\end{itemize}

\textbf{Usage:}
\begin{lstlisting}[language=Python, caption=Get Time Info Usage]
global_disc.set_time_parameters(dt=0.01, T=1.0)
time_info = global_disc.get_time_info()
print(f"Time step: {time_info['dt']}")
print(f"Final time: {time_info['T']}")
print(f"Number of time steps: {time_info['n_time_steps']}")
print(f"Time points shape: {time_info['time_points'].shape}")
\end{lstlisting}

\paragraph{get\_global\_info()}\leavevmode
\begin{lstlisting}[language=Python, caption=Get Global Info Method]
def get_global_info(self) -> dict
\end{lstlisting}

\textbf{Returns:} \texttt{dict} - Dictionary containing complete global discretization information

\textbf{Dictionary Keys:}
\begin{itemize}
    \item \texttt{n\_domains}: Number of domains
    \item \texttt{total\_elements}: Total number of elements across all domains
    \item \texttt{total\_nodes}: Total number of nodes across all domains
    \item \texttt{time\_info}: Time discretization information (from \texttt{get\_time\_info()})
    \item \texttt{spatial\_discretizations}: List of mesh information for each domain
\end{itemize}

\textbf{Usage:}
\begin{lstlisting}[language=Python, caption=Get Global Info Usage]
global_disc.set_time_parameters(dt=0.01, T=1.0)
global_info = global_disc.get_global_info()

print(f"Number of domains: {global_info['n_domains']}")
print(f"Total elements: {global_info['total_elements']}")
print(f"Total nodes: {global_info['total_nodes']}")

# Access time information
time_info = global_info['time_info']
print(f"Time step: {time_info['dt']}")

# Access spatial discretization info for each domain
for i, spatial_info in enumerate(global_info['spatial_discretizations']):
    print(f"Domain {i}: {spatial_info['n_elements']} elements")
\end{lstlisting}

\subsection{Complete Usage Examples}
\label{subsec:discretization_complete_examples}

\subsubsection{Single Domain Setup}

\begin{lstlisting}[language=Python, caption=Single Domain Discretization Setup]
from bionetflux.core.discretization import Discretization, GlobalDiscretization

# Create spatial discretization for a single domain
spatial_disc = Discretization(
    n_elements=40,
    domain_start=0.0,
    domain_length=1.0,
    stab_constant=1.0
)

# Set stabilization parameters (e.g., for Keller-Segel: 2 equations)
spatial_disc.set_tau([1.0, 1.0])

# Create global discretization
global_disc = GlobalDiscretization([spatial_disc])

# Set time parameters
global_disc.set_time_parameters(dt=0.01, T=1.0)

# Access mesh information
mesh_info = spatial_disc.get_mesh_info()
print(f"Elements: {mesh_info['n_elements']}")
print(f"Nodes: {mesh_info['n_nodes']}")
print(f"Element length: {mesh_info['element_length']}")

# Access global information
global_info = global_disc.get_global_info()
print(f"Total elements: {global_info['total_elements']}")
print(f"Time steps: {global_info['time_info']['n_time_steps']}")
\end{lstlisting}

\subsubsection{Multi-Domain Network Setup}

\begin{lstlisting}[language=Python, caption=Multi-Domain Network Discretization]
# Create multiple spatial discretizations for network domains
main_disc = Discretization(n_elements=30, domain_start=0.0, domain_length=1.0)
branch1_disc = Discretization(n_elements=20, domain_start=1.0, domain_length=0.8)
branch2_disc = Discretization(n_elements=20, domain_start=1.0, domain_length=0.8)

# Set stabilization parameters for each domain
main_disc.set_tau([1.0, 1.0, 1.0, 1.0])     # 4-equation system
branch1_disc.set_tau([1.0, 1.0, 1.0, 1.0])  # 4-equation system  
branch2_disc.set_tau([1.0, 1.0, 1.0, 1.0])  # 4-equation system

# Create global discretization for the network
network_global_disc = GlobalDiscretization([main_disc, branch1_disc, branch2_disc])

# Set global time parameters
network_global_disc.set_time_parameters(dt=0.01, T=0.5)

# Access individual domain discretizations
domain_0 = network_global_disc.get_spatial_discretization(0)
print(f"Main domain elements: {domain_0.n_elements}")

# Get global statistics
global_info = network_global_disc.get_global_info()
print(f"Network has {global_info['n_domains']} domains")
print(f"Total elements: {global_info['total_elements']}")
print(f"Total nodes: {global_info['total_nodes']}")
\end{lstlisting}

\subsection{Method Summary Tables}
\label{subsec:discretization_method_summary}

\subsubsection{Discretization Class Methods}

\begin{longtable}{|p{4.0cm}|p{2.5cm}|p{6.5cm}|}
\hline
\textbf{Method} & \textbf{Returns} & \textbf{Purpose} \\
\hline
\endhead

\texttt{\_generate\_mesh} & \texttt{None} & Generate spatial mesh (private, called during init) \\
\hline

\texttt{get\_mesh\_info} & \texttt{dict} & Return complete mesh information dictionary \\
\hline

\texttt{set\_tau} & \texttt{None} & Set stabilization parameters per equation \\
\hline

\end{longtable}

\subsubsection{GlobalDiscretization Class Methods}

\begin{longtable}{|p{5.2cm}|p{2.7cm}|p{6cm}|}
\hline
\textbf{Method} & \textbf{Returns} & \textbf{Purpose} \\
\hline
\endhead

\texttt{\_compute\_global\_info} & \texttt{None} & Compute global mesh info (private, called during init) \\
\hline

\texttt{set\_time\_parameters} & \texttt{None} & Set global time step and final time \\
\hline

\texttt{get\_spatial\_discretization} & \texttt{Discretization} & Access specific domain discretization \\
\hline

\texttt{get\_time\_info} & \texttt{dict} & Return time discretization information \\
\hline

\texttt{get\_global\_info} & \texttt{dict} & Return complete global discretization information \\
\hline

\end{longtable}

This documentation provides an exact reference for the Discretization module based on the actual BioNetFlux implementation in \texttt{bionetflux.core.discretization}.

\begin{todobox}
	\begin{itemize}
		\item Check doubling of information with geometry module
		\item Should the stabilization parameter be an attribute of the Discretization class? Probably not. Evaluate options. Modification has (positive) impact for {\tt static\_condensation}
		\item Allow for non uniform grids
		\item Allow for non uniform time stepping
		\item Include P\'eclet number checks and checks of requirements on {\tt dt} vs {\tt h}
		\end{itemize}
	\end{todobox}

% End of discretization module API documentation


\section{Discretization Module API Reference}
\label{sec:discretization_module_api}

This section provides a comprehensive reference for the Discretization classes (\texttt{bionetflux.core.discretization}) based on the actual implementation. The module contains classes for spatial and temporal discretization management using finite elements.

\subsection{Module Overview}

The discretization module contains two main classes:
\begin{itemize}
    \item \texttt{Discretization}: Single-domain spatial discretization using finite elements
    \item \texttt{GlobalDiscretization}: Multi-domain coordinator with time stepping parameters
\end{itemize}

\subsection{Module Dependencies}

\begin{lstlisting}[language=Python, caption=Module Dependencies]
import numpy as np
from typing import List, Optional
\end{lstlisting}

\subsection{Discretization Class}
\label{subsec:discretization_class}

The main class for single-domain spatial discretization using finite elements. Handles spatial mesh generation and element properties.

\subsubsection{Constructor}

\paragraph{\_\_init\_\_()}\leavevmode
\begin{lstlisting}[language=Python, caption=Discretization Constructor]
def __init__(self, n_elements: int, domain_start: float = 0.0, 
             domain_length: float = 1.0, stab_constant: float = 1.0)
\end{lstlisting}

\textbf{Parameters:}
\begin{itemize}
    \item \texttt{n\_elements}: Number of elements in the mesh
    \item \texttt{domain\_start}: Domain start coordinate (default: 0.0)
    \item \texttt{domain\_length}: Domain length (default: 1.0)
    \item \texttt{stab\_constant}: Stabilization constant (default: 1.0)
\end{itemize}

\textbf{Description:} Creates a spatial discretization and automatically generates the mesh by calling \texttt{\_generate\_mesh()}.

\textbf{Usage Examples:}
\begin{lstlisting}[language=Python, caption=Discretization Constructor Usage]
# Basic discretization with 20 elements
disc1 = Discretization(n_elements=20)

# Custom domain discretization
disc2 = Discretization(
    n_elements=40,
    domain_start=0.0,  
    domain_length=1.0,
    stab_constant=1.0
)

# Fine mesh discretization on different domain
disc3 = Discretization(
    n_elements=100,
    domain_start=-1.0,
    domain_length=2.0,
    stab_constant=0.5
)
\end{lstlisting}

\subsubsection{Core Attributes}

\begin{longtable}{|p{3.2cm}|p{3cm}|p{7cm}|}
\hline
\textbf{Attribute} & \textbf{Type} & \textbf{Description} \\
\hline
\endhead

\texttt{n\_elements} & \texttt{int} & Number of elements in the mesh \\
\hline

\texttt{domain\_start} & \texttt{float} & Start coordinate of the domain \\
\hline

\texttt{domain\_length} & \texttt{float} & Length of the domain \\
\hline

\texttt{stab\_constant} & \texttt{float} & Stabilization constant \\
\hline

\texttt{n\_nodes} & \texttt{int} & Number of nodes: \texttt{n\_elements + 1} \\
\hline

\texttt{element\_length} & \texttt{float} & Uniform element size: \texttt{domain\_length / n\_elements} \\
\hline

\texttt{nodes} & \texttt{np.ndarray} & Node coordinates array (generated by \texttt{\_generate\_mesh}) \\
\hline

\texttt{elements} & \texttt{np.ndarray} & Element connectivity array: \texttt{[[i, i+1] for i in range(n\_elements)]} \\
\hline

\texttt{element\_centers} & \texttt{np.ndarray} & Element center coordinates \\
\hline

\texttt{tau} & \texttt{np.ndarray} & Stabilization parameters per equation (set by \texttt{set\_tau}) \\
\hline

\end{longtable}

\subsubsection{Private Methods}

\paragraph{\_generate\_mesh()}\leavevmode
\begin{lstlisting}[language=Python, caption=Generate Mesh Method]
def _generate_mesh(self)
\end{lstlisting}

\textbf{Description:} Private method that generates the spatial mesh nodes and connectivity. Called automatically during initialization.

\textbf{Implementation:}
\begin{itemize}
    \item Creates \texttt{self.nodes} using \texttt{np.linspace(domain\_start, domain\_start + domain\_length, n\_nodes)}
    \item Creates \texttt{self.elements} as \texttt{np.array([[i, i+1] for i in range(n\_elements)])}
    \item Creates \texttt{self.element\_centers} as \texttt{nodes[:-1] + element\_length / 2}
\end{itemize}

\textbf{Usage:} This method is called automatically during object initialization and should not be called directly.

\subsubsection{Public Methods}

\paragraph{get\_mesh\_info()}\leavevmode
\begin{lstlisting}[language=Python, caption=Get Mesh Info Method]
def get_mesh_info(self) -> dict
\end{lstlisting}

\textbf{Returns:} \texttt{dict} - Dictionary containing complete mesh information

\textbf{Dictionary Keys:}
\begin{itemize}
    \item \texttt{n\_elements}: Number of elements
    \item \texttt{n\_nodes}: Number of nodes
    \item \texttt{element\_length}: Element length
    \item \texttt{nodes}: Node coordinates array
    \item \texttt{elements}: Element connectivity array
    \item \texttt{element\_centers}: Element center coordinates
    \item \texttt{domain\_start}: Domain start coordinate
    \item \texttt{domain\_length}: Domain length
    \item \texttt{stab\_constant}: Stabilization constant
\end{itemize}

\textbf{Usage:}
\begin{lstlisting}[language=Python, caption=Get Mesh Info Usage]
disc = Discretization(n_elements=10, domain_start=0.0, domain_length=2.0)
mesh_info = disc.get_mesh_info()
print(f"Elements: {mesh_info['n_elements']}")
print(f"Nodes: {mesh_info['n_nodes']}")
print(f"Element length: {mesh_info['element_length']}")
print(f"First few nodes: {mesh_info['nodes'][:5]}")
\end{lstlisting}

\paragraph{set\_tau()}\leavevmode
\begin{lstlisting}[language=Python, caption=Set Tau Method]
def set_tau(self, tau_values: List[float])
\end{lstlisting}

\textbf{Parameters:}
\begin{itemize}
    \item \texttt{tau\_values}: List of stabilization parameters for each equation
\end{itemize}

\textbf{Side Effects:} Sets \texttt{self.tau} as a numpy array

\textbf{Validation:} Raises \texttt{ValueError} if \texttt{tau\_values} list is empty

\textbf{Usage:}
\begin{lstlisting}[language=Python, caption=Set Tau Usage]
# Keller-Segel problem (2 equations)
ks_disc = Discretization(n_elements=20)
ks_disc.set_tau([1.0, 1.0])  # [tau_u, tau_phi]

# OrganOnChip problem (4 equations)
ooc_disc = Discretization(n_elements=40)
ooc_disc.set_tau([1.0, 1.0, 1.0, 1.0])  # [tu, to, tv, tp]

# Different stabilization parameters
custom_disc = Discretization(n_elements=30)
custom_disc.set_tau([0.5, 2.0, 1.5])  # Custom values per equation

# This will raise ValueError
try:
    disc.set_tau([])  # Empty list
except ValueError as e:
    print(f"Error: {e}")
\end{lstlisting}



\subsection{GlobalDiscretization Class}
\label{subsec:globaldiscretization_class}

Global discretization class managing multiple spatial domains and time discretization. Coordinates temporal evolution across all domains.

\subsubsection{Constructor}

\paragraph{\_\_init\_\_()}\leavevmode
\begin{lstlisting}[language=Python, caption=GlobalDiscretization Constructor]
def __init__(self, spatial_discretizations: List[Discretization])
\end{lstlisting}

\textbf{Parameters:}
\begin{itemize}
    \item \texttt{spatial\_discretizations}: List of Discretization instances for each domain
\end{itemize}

\textbf{Description:} Creates a global discretization coordinator and automatically computes global mesh information by calling \texttt{\_compute\_global\_info()}.

\textbf{Usage:}
\begin{lstlisting}[language=Python, caption=GlobalDiscretization Usage]
# Single domain
disc1 = Discretization(n_elements=20, domain_start=0.0, domain_length=1.0)
global_disc = GlobalDiscretization([disc1])

# Multi-domain network
main_disc = Discretization(n_elements=30, domain_start=0.0, domain_length=1.0)
branch1_disc = Discretization(n_elements=20, domain_start=1.0, domain_length=0.8)
branch2_disc = Discretization(n_elements=20, domain_start=1.0, domain_length=0.8)

multi_global_disc = GlobalDiscretization([main_disc, branch1_disc, branch2_disc])
\end{lstlisting}

\subsubsection{Core Attributes}

\begin{longtable}{|p{4.6cm}|p{4cm}|p{5.5cm}|}
\hline
\textbf{Attribute} & \textbf{Type} & \textbf{Description} \\
\hline
\endhead

\texttt{spatial\_discretizations} & \texttt{List[Discretization]} & List of spatial discretizations for each domain \\
\hline

\texttt{n\_domains} & \texttt{int} & Number of domains: \texttt{len(spatial\_discretizations)} \\
\hline

\texttt{dt} & \texttt{Optional[float]} & Global time step size (initially None) \\
\hline

\texttt{T} & \texttt{Optional[float]} & Final time (initially None) \\
\hline

\texttt{n\_time\_steps} & \texttt{Optional[int]} & Number of time steps: \texttt{int(np.ceil(T/dt))} \\
\hline

\texttt{time\_points} & \texttt{Optional[np.ndarray]} & Array of time points from 0 to T \\
\hline

\texttt{total\_elements} & \texttt{int} & Sum of elements across all domains \\
\hline

\texttt{total\_nodes} & \texttt{int} & Sum of nodes across all domains \\
\hline

\end{longtable}

\subsubsection{Private Methods}

\paragraph{\_compute\_global\_info()}\leavevmode
\begin{lstlisting}[language=Python, caption=Compute Global Info Method]
def _compute_global_info(self)
\end{lstlisting}

\textbf{Description:} Private method that computes global mesh information from all domains. Called automatically during initialization.

\textbf{Implementation:}
\begin{itemize}
    \item Sets \texttt{self.total\_elements} as sum of elements from all spatial discretizations
    \item Sets \texttt{self.total\_nodes} as sum of nodes from all spatial discretizations
\end{itemize}

\textbf{Usage:} This method is called automatically during object initialization and should not be called directly.

\subsubsection{Public Methods}

\paragraph{set\_time\_parameters()}\leavevmode
\begin{lstlisting}[language=Python, caption=Set Time Parameters Method]
def set_time_parameters(self, dt: float, T: float)
\end{lstlisting}

\textbf{Parameters:}
\begin{itemize}
    \item \texttt{dt}: Time step size
    \item \texttt{T}: Final time
\end{itemize}

\textbf{Side Effects:} 
\begin{itemize}
    \item Sets \texttt{self.dt}, \texttt{self.T}, \texttt{self.n\_time\_steps}, and \texttt{self.time\_points}
    \item Calculates \texttt{n\_time\_steps = int(np.ceil(T / dt))}
    \item Creates \texttt{time\_points = np.linspace(0, T, n\_time\_steps + 1)}
\end{itemize}

\textbf{Usage:}
\begin{lstlisting}[language=Python, caption=Set Time Parameters Usage]
global_disc = GlobalDiscretization([disc1, disc2])
global_disc.set_time_parameters(dt=0.01, T=1.0)

print(f"Time step: {global_disc.dt}")
print(f"Final time: {global_disc.T}")
print(f"Number of time steps: {global_disc.n_time_steps}")
print(f"First few time points: {global_disc.time_points[:5]}")
\end{lstlisting}

\paragraph{get\_spatial\_discretization()}\leavevmode
\begin{lstlisting}[language=Python, caption=Get Spatial Discretization Method]
def get_spatial_discretization(self, domain_index: int) -> Discretization
\end{lstlisting}

\textbf{Parameters:}
\begin{itemize}
    \item \texttt{domain\_index}: Domain index (0 to \texttt{n\_domains-1})
\end{itemize}

\textbf{Returns:} \texttt{Discretization} - Spatial discretization for specified domain

\textbf{Raises:} \texttt{IndexError} if domain\_index is out of range

\textbf{Usage:}
\begin{lstlisting}[language=Python, caption=Get Spatial Discretization Usage]
try:
    domain_0_disc = global_disc.get_spatial_discretization(0)
    print(f"Domain 0 elements: {domain_0_disc.n_elements}")
    print(f"Domain 0 nodes: {domain_0_disc.n_nodes}")
    
    # This will raise IndexError if only 2 domains exist
    domain_5_disc = global_disc.get_spatial_discretization(5)
except IndexError as e:
    print(f"Error: {e}")
\end{lstlisting}

\paragraph{get\_time\_info()}\leavevmode
\begin{lstlisting}[language=Python, caption=Get Time Info Method]
def get_time_info(self) -> dict
\end{lstlisting}

\textbf{Returns:} \texttt{dict} - Dictionary containing time discretization information

\textbf{Dictionary Keys:}
\begin{itemize}
    \item \texttt{dt}: Time step size
    \item \texttt{T}: Final time
    \item \texttt{n\_time\_steps}: Number of time steps
    \item \texttt{time\_points}: Array of time points
\end{itemize}

\textbf{Usage:}
\begin{lstlisting}[language=Python, caption=Get Time Info Usage]
global_disc.set_time_parameters(dt=0.01, T=1.0)
time_info = global_disc.get_time_info()
print(f"Time step: {time_info['dt']}")
print(f"Final time: {time_info['T']}")
print(f"Number of time steps: {time_info['n_time_steps']}")
print(f"Time points shape: {time_info['time_points'].shape}")
\end{lstlisting}

\paragraph{get\_global\_info()}\leavevmode
\begin{lstlisting}[language=Python, caption=Get Global Info Method]
def get_global_info(self) -> dict
\end{lstlisting}

\textbf{Returns:} \texttt{dict} - Dictionary containing complete global discretization information

\textbf{Dictionary Keys:}
\begin{itemize}
    \item \texttt{n\_domains}: Number of domains
    \item \texttt{total\_elements}: Total number of elements across all domains
    \item \texttt{total\_nodes}: Total number of nodes across all domains
    \item \texttt{time\_info}: Time discretization information (from \texttt{get\_time\_info()})
    \item \texttt{spatial\_discretizations}: List of mesh information for each domain
\end{itemize}

\textbf{Usage:}
\begin{lstlisting}[language=Python, caption=Get Global Info Usage]
global_disc.set_time_parameters(dt=0.01, T=1.0)
global_info = global_disc.get_global_info()

print(f"Number of domains: {global_info['n_domains']}")
print(f"Total elements: {global_info['total_elements']}")
print(f"Total nodes: {global_info['total_nodes']}")

# Access time information
time_info = global_info['time_info']
print(f"Time step: {time_info['dt']}")

# Access spatial discretization info for each domain
for i, spatial_info in enumerate(global_info['spatial_discretizations']):
    print(f"Domain {i}: {spatial_info['n_elements']} elements")
\end{lstlisting}

\subsection{Complete Usage Examples}
\label{subsec:discretization_complete_examples}

\subsubsection{Single Domain Setup}

\begin{lstlisting}[language=Python, caption=Single Domain Discretization Setup]
from bionetflux.core.discretization import Discretization, GlobalDiscretization

# Create spatial discretization for a single domain
spatial_disc = Discretization(
    n_elements=40,
    domain_start=0.0,
    domain_length=1.0,
    stab_constant=1.0
)

# Set stabilization parameters (e.g., for Keller-Segel: 2 equations)
spatial_disc.set_tau([1.0, 1.0])

# Create global discretization
global_disc = GlobalDiscretization([spatial_disc])

# Set time parameters
global_disc.set_time_parameters(dt=0.01, T=1.0)

# Access mesh information
mesh_info = spatial_disc.get_mesh_info()
print(f"Elements: {mesh_info['n_elements']}")
print(f"Nodes: {mesh_info['n_nodes']}")
print(f"Element length: {mesh_info['element_length']}")

# Access global information
global_info = global_disc.get_global_info()
print(f"Total elements: {global_info['total_elements']}")
print(f"Time steps: {global_info['time_info']['n_time_steps']}")
\end{lstlisting}

\subsubsection{Multi-Domain Network Setup}

\begin{lstlisting}[language=Python, caption=Multi-Domain Network Discretization]
# Create multiple spatial discretizations for network domains
main_disc = Discretization(n_elements=30, domain_start=0.0, domain_length=1.0)
branch1_disc = Discretization(n_elements=20, domain_start=1.0, domain_length=0.8)
branch2_disc = Discretization(n_elements=20, domain_start=1.0, domain_length=0.8)

# Set stabilization parameters for each domain
main_disc.set_tau([1.0, 1.0, 1.0, 1.0])     # 4-equation system
branch1_disc.set_tau([1.0, 1.0, 1.0, 1.0])  # 4-equation system  
branch2_disc.set_tau([1.0, 1.0, 1.0, 1.0])  # 4-equation system

# Create global discretization for the network
network_global_disc = GlobalDiscretization([main_disc, branch1_disc, branch2_disc])

# Set global time parameters
network_global_disc.set_time_parameters(dt=0.01, T=0.5)

# Access individual domain discretizations
domain_0 = network_global_disc.get_spatial_discretization(0)
print(f"Main domain elements: {domain_0.n_elements}")

# Get global statistics
global_info = network_global_disc.get_global_info()
print(f"Network has {global_info['n_domains']} domains")
print(f"Total elements: {global_info['total_elements']}")
print(f"Total nodes: {global_info['total_nodes']}")
\end{lstlisting}

\subsection{Method Summary Tables}
\label{subsec:discretization_method_summary}

\subsubsection{Discretization Class Methods}

\begin{longtable}{|p{4.0cm}|p{2.5cm}|p{6.5cm}|}
\hline
\textbf{Method} & \textbf{Returns} & \textbf{Purpose} \\
\hline
\endhead

\texttt{\_generate\_mesh} & \texttt{None} & Generate spatial mesh (private, called during init) \\
\hline

\texttt{get\_mesh\_info} & \texttt{dict} & Return complete mesh information dictionary \\
\hline

\texttt{set\_tau} & \texttt{None} & Set stabilization parameters per equation \\
\hline

\end{longtable}

\subsubsection{GlobalDiscretization Class Methods}

\begin{longtable}{|p{5.2cm}|p{2.7cm}|p{6cm}|}
\hline
\textbf{Method} & \textbf{Returns} & \textbf{Purpose} \\
\hline
\endhead

\texttt{\_compute\_global\_info} & \texttt{None} & Compute global mesh info (private, called during init) \\
\hline

\texttt{set\_time\_parameters} & \texttt{None} & Set global time step and final time \\
\hline

\texttt{get\_spatial\_discretization} & \texttt{Discretization} & Access specific domain discretization \\
\hline

\texttt{get\_time\_info} & \texttt{dict} & Return time discretization information \\
\hline

\texttt{get\_global\_info} & \texttt{dict} & Return complete global discretization information \\
\hline

\end{longtable}

This documentation provides an exact reference for the Discretization module based on the actual BioNetFlux implementation in \texttt{bionetflux.core.discretization}.

\begin{todobox}
	\begin{itemize}
		\item Check doubling of information with geometry module
		\item Should the stabilization parameter be an attribute of the Discretization class? Probably not. Evaluate options. Modification has (positive) impact for {\tt static\_condensation}
		\item Allow for non uniform grids
		\item Allow for non uniform time stepping
		\item Include P\'eclet number checks and checks of requirements on {\tt dt} vs {\tt h}
		\end{itemize}
	\end{todobox}

% End of discretization module API documentation

\input{bulk_data_module_api}


\section{Domain Data Module}
\label{sec:domain_data_module}

The \texttt{domain\_data} module provides a lightweight container for storing essential domain-specific information extracted from framework objects. This module implements a simple data storage pattern that avoids holding references to full problem and discretization objects, instead extracting and storing only the essential data needed for bulk operations in the HDG method.

\subsection{Overview}

The \texttt{DomainData} class serves as a simple data container that encapsulates extracted domain-specific information. The primary design goal is to minimize memory usage by storing only essential extracted data rather than maintaining references to complete framework objects. This approach enables efficient bulk operations while reducing memory overhead.

\subsection{Class Definition}

\subsubsection{DomainData Class}

The \texttt{DomainData} class is implemented as a simple container class with manual initialization:

\begin{lstlisting}[language=Python, caption=DomainData Class Definition]
	class DomainData:
	"""
	Lightweight container storing only essential extracted data for a domain.
	
	This avoids storing full problem/discretization objects and extracts
	only what's needed for bulk operations.
	"""
\end{lstlisting}

\subsubsection{Constructor}

The class provides a single constructor that accepts all required data:

\begin{lstlisting}[language=Python, caption=Constructor Signature]
	def __init__(self,
	neq: int,
	n_elements: int,
	nodes: np.ndarray,
	element_length: float,
	mass_matrix: np.ndarray,
	trace_matrix: np.ndarray,
	initial_conditions: List[Optional[Callable]],
	forcing_functions: List[Optional[Callable]]):
\end{lstlisting}

\subsubsection{Constructor Parameters}

\begin{itemize}
	\item \textbf{neq} (\texttt{int}): Number of coupled equations in the system
	\item \textbf{n\_elements} (\texttt{int}): Total number of finite elements in the domain
	\item \textbf{nodes} (\texttt{np.ndarray}): Array containing node coordinates
	\item \textbf{element\_length} (\texttt{float}): Length of individual elements (assuming uniform discretization)
	\item \textbf{mass\_matrix} (\texttt{np.ndarray}): 2×2 mass matrix extracted from static condensation
	\item \textbf{trace\_matrix} (\texttt{np.ndarray}): Trace matrix extracted from static condensation  
	\item \textbf{initial\_conditions} (\texttt{List[Optional[Callable]]}): List of initial condition functions for each equation
	\item \textbf{forcing\_functions} (\texttt{List[Optional[Callable]]}): List of forcing functions for each equation
\end{itemize}

\subsection{Instance Attributes}

After initialization, the \texttt{DomainData} object stores the following attributes:

\subsubsection{Discretization Parameters}
\begin{itemize}
	\item \textbf{self.neq}: Number of equations (copied from input)
	\item \textbf{self.n\_elements}: Number of elements (copied from input)  
	\item \textbf{self.nodes}: Node coordinates array (deep copied via \texttt{nodes.copy()})
	\item \textbf{self.element\_length}: Element length (copied from input)
\end{itemize}

\subsubsection{Matrix Data}
\begin{itemize}
	\item \textbf{self.mass\_matrix}: Mass matrix (deep copied via \texttt{mass\_matrix.copy()})
	\item \textbf{self.trace\_matrix}: Trace matrix (deep copied via \texttt{trace\_matrix.copy()})
\end{itemize}

\subsubsection{Function Lists}
\begin{itemize}
	\item \textbf{self.initial\_conditions}: Initial condition functions (shallow copied via \texttt{initial\_conditions.copy()})
	\item \textbf{self.forcing\_functions}: Forcing functions (shallow copied via \texttt{forcing\_functions.copy()})
\end{itemize}

\subsection{Methods}

The \texttt{DomainData} class provides minimal functionality with only essential methods:

\subsubsection{String Representation}

\begin{lstlisting}[language=Python, caption=String Representation Method]
	def __str__(self) -> str:
	return (f"DomainData(neq={self.neq}, n_elements={self.n_elements}, "
	f"element_length={self.element_length})")
\end{lstlisting}

This method provides a concise string representation showing the key discretization parameters: number of equations, number of elements, and element length.

\subsection{Data Management Strategy}

\subsubsection{Memory Management}

The class implements a careful data copying strategy:

\begin{itemize}
	\item \textbf{Deep Copy for Arrays}: Numerical arrays (\texttt{nodes}, \texttt{mass\_matrix}, \texttt{trace\_matrix}) are deep copied using \texttt{.copy()} to prevent unintended modifications
	\item \textbf{Shallow Copy for Function Lists}: Function lists are shallow copied, meaning the list structure is copied but the function objects themselves are shared references
	\item \textbf{Direct Copy for Scalars}: Scalar values (\texttt{neq}, \texttt{n\_elements}, \texttt{element\_length}) are directly assigned
\end{itemize}

\subsubsection{Data Extraction Pattern}

The \texttt{DomainData} class follows an extraction pattern where essential data is pulled from complex framework objects and stored in a simplified container. This approach provides:

\begin{itemize}
	\item \textbf{Reduced Memory Footprint}: Avoids storing references to large framework objects
	\item \textbf{Simplified Access}: Direct attribute access without method calls
	\item \textbf{Data Independence}: Extracted data remains valid even if source objects are modified or destroyed
\end{itemize}

\subsection{Usage Patterns}

\subsubsection{Typical Construction}

\texttt{DomainData} objects are typically created by extraction methods rather than direct construction:

\begin{lstlisting}[language=Python, caption=Typical Usage Pattern]
	# Usually created by extraction utilities, not directly
	domain_data = DomainData(
	neq=problem.neq,
	n_elements=discretization.n_elements, 
	nodes=discretization.nodes,
	element_length=discretization.element_length,
	mass_matrix=static_condensation_matrices['M'],
	trace_matrix=static_condensation_matrices['T'],
	initial_conditions=[problem.u0[i] for i in range(problem.neq)],
	forcing_functions=[problem.force[i] for i in range(problem.neq)]
	)
\end{lstlisting}

\subsubsection{Data Access}

Once created, data is accessed through simple attribute access:

\begin{lstlisting}[language=Python, caption=Data Access Pattern]
	# Direct attribute access
	num_equations = domain_data.neq
	num_elements = domain_data.n_elements
	node_coordinates = domain_data.nodes
	mass_matrix = domain_data.mass_matrix
	
	# Function access
	if domain_data.initial_conditions[0] is not None:
	initial_value = domain_data.initial_conditions[0](x=0.5, t=0.0)
\end{lstlisting}

\subsection{Integration Context}

While the \texttt{DomainData} class itself is simple, it serves as a foundational component in the larger framework:

\subsubsection{Framework Integration}

The class is primarily used by other components such as:

\begin{itemize}
	\item \textbf{BulkDataManager}: Uses \texttt{DomainData} objects to store extracted domain information
	\item \textbf{Static Condensation}: Accesses matrices stored in \texttt{DomainData}
	\item \textbf{Time Integration}: Uses initial conditions and forcing functions
\end{itemize}

\subsubsection{Collection Processing}

\texttt{DomainData} objects are typically processed in collections for multi-domain problems:

\begin{lstlisting}[language=Python, caption=Collection Processing]
	# Typical usage in collections
	domain_data_list = [...]  # List of DomainData objects
	
	for i, domain_data in enumerate(domain_data_list):
	print(f"Domain {i}: {domain_data}")  # Uses __str__ method
	
	# Process domain-specific data
	process_domain_matrices(domain_data.mass_matrix, domain_data.trace_matrix)
	evaluate_initial_conditions(domain_data.initial_conditions, t=0.0)
\end{lstlisting}

\subsection{Design Characteristics}

\subsubsection{Simplicity}

The \texttt{DomainData} class embodies a minimalist design philosophy:

\begin{itemize}
	\item \textbf{No Methods}: Apart from \texttt{\_\_str\_\_}, the class contains no computational methods
	\item \textbf{Direct Storage}: All data is stored as instance attributes
	\item \textbf{No Inheritance}: Simple class with no base classes or complex hierarchies
	\item \textbf{No Properties}: Direct attribute access without getter/setter methods
\end{itemize}

\subsubsection{Data Integrity}

While the class doesn't enforce immutability, it provides data integrity through copying:

\begin{itemize}
	\item \textbf{Independent Arrays}: Numerical arrays are independent of source objects
	\item \textbf{Safe Modification}: Changes to source objects don't affect stored data
	\item \textbf{Controlled Access}: Data access requires explicit attribute references
\end{itemize}

\subsection{Limitations and Considerations}

\subsubsection{Mutability}

The class does not enforce immutability, meaning stored data can be modified after creation:

\begin{lstlisting}[language=Python, caption=Mutability Consideration]
	domain_data = DomainData(...)
	
	# These modifications are possible but not recommended
	domain_data.neq = 5  # Modifies equation count
	domain_data.mass_matrix[0, 0] = 1.0  # Modifies matrix element
\end{lstlisting}

\subsubsection{Type Safety}

The class provides type hints but no runtime type checking:

\begin{itemize}
	\item \textbf{Documentation}: Type hints serve as documentation
	\item \textbf{Static Analysis}: Enables static analysis tools to detect type issues
	\item \textbf{Runtime Safety}: No runtime validation of input types or values
\end{itemize}

\subsection{Summary}

The \texttt{domain\_data} module provides a straightforward, lightweight container for domain-specific data extraction. Key characteristics include:

\begin{itemize}
	\item \textbf{Minimal Design}: Simple class with direct attribute storage
	\item \textbf{Memory Efficiency}: Stores only essential extracted data
	\item \textbf{Data Independence}: Copied data remains valid regardless of source object state
	\item \textbf{Framework Integration}: Serves as foundational component for bulk operations
	\item \textbf{Type Documentation}: Comprehensive type hints for all parameters and attributes
\end{itemize}

This module represents a clean implementation of the data extraction pattern, providing the essential functionality needed for efficient bulk operations in the HDG framework while maintaining simplicity and clarity in its interface and implementation.

\begin{todobox}
	\begin{itemize}
		\item Evaluate the possibility of transforming the class in a {\tt @dataclass} object
		\item Merge file with {\tt BulkDataManager}
	\end{itemize}
\end{todobox}




% Lean Bulk Data Manager Module API Documentation (Accurate Analysis)
% To be included in master LaTeX document
%
% Usage: % Lean Bulk Data Manager Module API Documentation (Accurate Analysis)
% To be included in master LaTeX document
%
% Usage: % Lean Bulk Data Manager Module API Documentation (Accurate Analysis)
% To be included in master LaTeX document
%
% Usage: \input{docs/lean_bulk_data_manager_api}

\section{Lean Bulk Data Manager Module API Reference}
\label{sec:lean_bulk_data_manager_api}

This section provides an exact reference for the BulkDataManager class  based on detailed analysis of  the  \texttt{ooc1d.core.lean\_bulk\_data\_manager.BulkDataManager} module. This is an ultra-lean coordinator that minimizes memory usage by storing only essential domain data and accepting framework objects as method parameters.

\subsection{Module Overview}

The lean bulk data manager provides memory-efficient coordination for bulk operations by:
\begin{itemize}
    \item Storing only extracted essential domain data
    \item Accepting framework objects as method parameters
    \item Validating framework object compatibility
    \item Providing flexible bulk data operations without memory overhead
\end{itemize}

\subsection{Module Imports and Dependencies}

\begin{lstlisting}[language=Python, caption=Module Dependencies]
import numpy as np
from typing import List, Optional, Callable
from ooc1d.core.bulk_data import BulkData
from ooc1d.core.domain_data import DomainData
\end{lstlisting}

\subsection{BulkDataManager Class Definition}
\label{subsec:lean_bulk_data_manager_class}

\begin{lstlisting}[language=Python, caption=Class Declaration]
class BulkDataManager:
    """
    Ultra-lean coordinator for bulk operations in HDG method.
    
    This class stores only essential extracted domain data and accepts
    framework objects as parameters to methods that need them. This approach
    minimizes memory usage and increases flexibility.
    """
\end{lstlisting}

\subsection{Constructor}
\label{subsec:lean_constructor}

\paragraph{\_\_init\_\_()}\leavevmode
\begin{lstlisting}[language=Python, caption=Lean BulkDataManager Constructor]
def __init__(self, domain_data_list: List[DomainData])
\end{lstlisting}

\textbf{Parameters:}
\begin{itemize}
    \item \texttt{domain\_data\_list}: List of DomainData objects with essential extracted information
\end{itemize}

\textbf{Side Effects:} Sets \texttt{self.domain\_data\_list} attribute

\textbf{Usage:}
\begin{lstlisting}[language=Python, caption=Constructor Usage]
# Extract domain data first (see static factory method)
domain_data_list = BulkDataManager.extract_domain_data_list(
    problems, discretizations, static_condensations
)

# Create lean manager with extracted data only
lean_manager = BulkDataManager(domain_data_list)
\end{lstlisting}

\subsection{Core Attributes}
\label{subsec:lean_attributes}

\begin{longtable}{|p{3.5cm}|p{3.5cm}|p{6cm}|}
\hline
\textbf{Attribute} & \textbf{Type} & \textbf{Description} \\
\hline
\endhead

\texttt{domain\_data\_list} & \texttt{List[DomainData]} & List of essential domain data objects (only stored attribute) \\
\hline

\end{longtable}

\textbf{Memory Efficiency:} The lean manager stores \textbf{only} the domain data list, avoiding storage of large framework objects.

\subsection{Validation Methods}
\label{subsec:validation_methods}

\paragraph{\_validate\_framework\_objects()}\leavevmode
\begin{lstlisting}[language=Python, caption=Framework Validation Method]
def _validate_framework_objects(self, 
                               problems: List = None,
                               discretizations: List = None, 
                               static_condensations: List = None,
                               operation_name: str = "operation") -> None
\end{lstlisting}

\textbf{Parameters:}
\begin{itemize}
    \item \texttt{problems}: List of Problem objects to validate (optional)
    \item \texttt{discretizations}: List of discretization objects to validate (optional)
    \item \texttt{static\_condensations}: List of static condensation objects to validate (optional)
    \item \texttt{operation\_name}: Name of operation for error messages (default: ``operation")
\end{itemize}

\textbf{Returns:} \texttt{None}

\textbf{Raises:} \texttt{ValueError} for incompatible framework objects

\textbf{Validation Checks:}
\begin{enumerate}
    \item \textbf{List Length Validation}: All provided lists must match domain count
    \item \textbf{Problem Validation}: \texttt{neq} attribute must match stored domain data
    \item \textbf{Discretization Validation}: \texttt{n\_elements}, nodes, and \texttt{element\_length} compatibility
    \item \textbf{Static Condensation Validation}: Matrix compatibility and method availability
\end{enumerate}

\textbf{Usage:}
\begin{lstlisting}[language=Python, caption=Validation Usage]
try:
    lean_manager._validate_framework_objects(
        problems=problems,
        discretizations=discretizations,
        operation_name="my_operation"
    )
    print("✓ Framework objects are compatible")
except ValueError as e:
    print(f"✗ Validation failed: {e}")
\end{lstlisting}

\subsection{Static Factory Methods}
\label{subsec:static_factory_methods}

\paragraph{extract\_domain\_data\_list()}\leavevmode
\begin{lstlisting}[language=Python, caption=Domain Data Extraction Method]
@staticmethod
def extract_domain_data_list(problems: List, 
                            discretizations: List, 
                            static_condensations: List) -> List[DomainData]
\end{lstlisting}

\textbf{Parameters:}
\begin{itemize}
    \item \texttt{problems}: List of Problem instances
    \item \texttt{discretizations}: List of discretization instances
    \item \texttt{static\_condensations}: List of static condensation instances
\end{itemize}

\textbf{Returns:} \texttt{List[DomainData]} - Extracted essential domain information

\textbf{Purpose:} Static factory method to extract and store essential data once for reuse

\textbf{Usage:}
\begin{lstlisting}[language=Python, caption=Domain Data Extraction Usage]
# One-time extraction of essential domain data
domain_data_list = BulkDataManager.extract_domain_data_list(
    problems=problems,
    discretizations=discretizations,
    static_condensations=static_condensations
)

# Can create multiple lean managers with same extracted data
lean_manager_1 = BulkDataManager(domain_data_list)
lean_manager_2 = BulkDataManager(domain_data_list)
\end{lstlisting}

\paragraph{\_extract\_single\_domain\_data()}\leavevmode
\begin{lstlisting}[language=Python, caption=Single Domain Extraction Method]
@staticmethod
def _extract_single_domain_data(problem, discretization, sc, domain_idx: int) -> DomainData
\end{lstlisting}

\textbf{Parameters:}
\begin{itemize}
    \item \texttt{problem}: Problem object for single domain
    \item \texttt{discretization}: Discretization object for single domain
    \item \texttt{sc}: Static condensation object for single domain
    \item \texttt{domain\_idx}: Domain index for error reporting
\end{itemize}

\textbf{Returns:} \texttt{DomainData} - Extracted essential data for single domain

\textbf{Extraction Process:}
\begin{enumerate}
    \item Extract matrices from static condensation (\texttt{M}, \texttt{T})
    \item Extract initial conditions with multiple access pattern support
    \item Extract forcing functions with multiple access pattern support
    \item Create DomainData object with essential information
\end{enumerate}

\subsection{BulkData Creation Methods}
\label{subsec_bulk_data_creation}

Bulk data can be dual ({\tt dual = True}) or primal ({\tt dual = False}). When representing a physical object $w$ as a primal bulk data object, the two data points $w_0$ and $w_1$ relative to the element $E$ are the values of the degrees of freedom such that on $E$ we have \(w = \sum_{i=0}^1 w_i e_i\), where $e_0$, $e_1$ are the chosen basis for the P1 space on $E$. For the nodal basis, $w_0$ and $w_1$ are the values of $w$ at the two vertices of $E$. When representing $w$ with a dual bulk object, $w_0$ and $w_1$ are the actions of $w$ on the basis functions: \(w_i = \int_E w e_i\). The choice on whether to use a primal or a dual bulk data object to represent a physical function, depends on its role in the problem/implementation. The solution will typically be represented by a primal bulk data object, while the right hand side will be represented by a dual object. The {\tt BulkData} creation methods handles these possibilities. 


\paragraph{create\_bulk\_data()}\leavevmode
\begin{lstlisting}[language=Python, caption=Create BulkData Method]
def create_bulk_data(self, 
                    domain_index: int, 
                    problem, 
                    discretization, 
                    dual: bool = False) -> BulkData
\end{lstlisting}

\textbf{Parameters:}
\begin{itemize}
    \item \texttt{domain\_index}: Index of the domain (0 to \texttt{n\_domains-1})
    \item \texttt{problem}: Problem object for this domain
    \item \texttt{discretization}: Discretization object for this domain
    \item \texttt{dual}: Whether to use dual formulation (default: False)
\end{itemize}

\textbf{Returns:} \texttt{BulkData} - New BulkData object

\textbf{Raises:} \texttt{ValueError} for invalid domain index

\textbf{Usage:}
\begin{lstlisting}[language=Python, caption=BulkData Creation Usage]
# Create primal BulkData for domain 0
bulk_data_primal = lean_manager.create_bulk_data(
    domain_index=0,
    problem=problems[0],
    discretization=discretizations[0],
    dual=False
)

# Create dual BulkData for forcing terms
bulk_data_dual = lean_manager.create_bulk_data(
    domain_index=0,
    problem=problems[0],
    discretization=discretizations[0],
    dual=True
)
\end{lstlisting}

\subsection{Bulk Operations Methods}
\label{subsec:bulk_operations}

\paragraph{compute\_source\_terms()}\leavevmode
\begin{lstlisting}[language=Python, caption=Compute Source Terms Method]
def compute_source_terms(self,
                        problems: List,
                        discretizations: List,
                        time: float) -> List[BulkData]
\end{lstlisting}


\textbf{Parameters:}
\begin{itemize}
    \item \texttt{problems}: List of Problem objects
    \item \texttt{discretizations}: List of discretization objects
    \item \texttt{time}: Current time for evaluation
\end{itemize}

\textbf{Returns:} \texttt{List[BulkData]} - Source terms for all domains using dual formulation

\textbf{Process:}
\begin{enumerate}
    \item Validate framework objects against stored domain data
    \item Create dual BulkData for each domain
    \item Set forcing functions from problems at specified time
    \item Return list of integrated source terms
\end{enumerate}

\textbf{Usage:}
\begin{lstlisting}[language=Python, caption=Source Terms Usage]
# Compute source terms at time t=0.5
source_terms = lean_manager.compute_source_terms(
    problems=problems,
    discretizations=discretizations,
    time=0.5
)

print(f"Computed {len(source_terms)} source term BulkData objects")
\end{lstlisting}

\paragraph{compute\_forcing\_terms()}\leavevmode
\begin{lstlisting}[language=Python, caption=Compute Forcing Terms Method]
def compute_forcing_terms(self, 
                          bulk_data_list: List[BulkData],
                          problems: List,
                          discretizations: List, 
                          time: float, 
                          dt: float) -> List[np.ndarray]
\end{lstlisting}

\textbf{Parameters:}
\begin{itemize}
    \item \texttt{bulk\_data\_list}: List of current BulkData solutions
    \item \texttt{problems}: List of Problem objects
    \item \texttt{discretizations}: List of discretization objects
    \item \texttt{time}: Current time
    \item \texttt{dt}: Time step size
\end{itemize}

\textbf{Returns:} \texttt{List[np.ndarray]} - Forcing term arrays (dual) for implicit Euler

\textbf{Computation:} For each domain: \texttt{forcing\_term = dt * force\_contrib + M * U\_old}

\textbf{Usage:}
\begin{lstlisting}[language=Python, caption=Forcing Terms Usage]
# Compute forcing terms for implicit Euler step
current_bulk_data = [...]  # Current solution
forcing_terms = lean_manager.compute_forcing_terms(
    bulk_data_list=current_bulk_data,
    problems=problems,
    discretizations=discretizations,
    time=0.1,
    dt=0.01
)

# forcing_terms[i] has shape (2*neq, n_elements) for domain i
\end{lstlisting}

\subsection{Initialization Methods}
\label{subsec:initialization_methods}

\paragraph{initialize\_all\_bulk\_data()}\leavevmode
\begin{lstlisting}[language=Python, caption=Initialize All BulkData Method]
def initialize_all_bulk_data(self, 
                            problems: List,
                            discretizations: List,
                            time: float = 0.0) -> List[BulkData]
\end{lstlisting}

\textbf{Parameters:}
\begin{itemize}
    \item \texttt{problems}: List of Problem objects
    \item \texttt{discretizations}: List of discretization objects
    \item \texttt{time}: Initial time (default: 0.0)
\end{itemize}

\textbf{Returns:} \texttt{List[BulkData]} - Initialized BulkData objects containing primal representation of initial conditions for all domains; calls {\tt initialize\_bulk\_data\_from\_initial\_conditions} for all domains

\textbf{Usage:}
\begin{lstlisting}[language=Python, caption=Initialize All Usage]
# Initialize all domains with initial conditions
initial_bulk_data = lean_manager.initialize_all_bulk_data(
    problems=problems,
    discretizations=discretizations,
    time=0.0
)

print(f"Initialized {len(initial_bulk_data)} BulkData objects")
\end{lstlisting}

\paragraph{initialize\_bulk\_data\_from\_initial\_conditions()}\leavevmode
\begin{lstlisting}[language=Python, caption=Initialize Single Domain Method]
def initialize_bulk_data_from_initial_conditions(self, 
                                                domain_index: int,
                                                problem,
                                                discretization,
                                                time: float = 0.0) -> BulkData
\end{lstlisting}

\textbf{Parameters:}
\begin{itemize}
    \item \texttt{domain\_index}: Index of domain to initialize
    \item \texttt{problem}: Problem object for this domain
    \item \texttt{discretization}: Discretization object for this domain
    \item \texttt{time}: Initial time (default: 0.0)
\end{itemize}

\textbf{Returns:} \texttt{BulkData} - Initialized BulkData object containing primal representation of initial conditions

\textbf{Process:}
\begin{enumerate}
    \item Validate domain index and framework objects
    \item Create BulkData with primal formulation
    \item Set initial conditions from stored domain data
    \item Default to zero if no initial conditions available
\end{enumerate}

\textbf{Usage:}
\begin{lstlisting}[language=Python, caption=Initialize Single Domain Usage]
# Initialize specific domain
domain_0_bulk = lean_manager.initialize_bulk_data_from_initial_conditions(
    domain_index=0,
    problem=problems[0],
    discretization=discretizations[0],
    time=0.0
)
\end{lstlisting}

\subsection{Data Management Methods}
\label{subsec:data_management}

\paragraph{update\_bulk\_data()}\leavevmode
\begin{lstlisting}[language=Python, caption=Update BulkData Method]
def update_bulk_data(self, bulk_data_list: List[BulkData], new_data_list: List[np.ndarray])
\end{lstlisting}

\textbf{Parameters:}
\begin{itemize}
    \item \texttt{bulk\_data\_list}: List of BulkData objects to update
    \item \texttt{new\_data\_list}: List of new bulk solution arrays
\end{itemize}

\textbf{Returns:} \texttt{None}

\textbf{Side Effects:} Updates data in all BulkData objects

\textbf{Validation:}
\begin{itemize}
    \item Lists must have matching lengths
    \item New data must have compatible shapes
    \item New data must not contain NaN or infinite values
\end{itemize}

\textbf{Usage:}
\begin{lstlisting}[language=Python, caption=Update BulkData Usage]
# Update bulk data with new solution
new_solutions = [...]  # List of numpy arrays
lean_manager.update_bulk_data(
    bulk_data_list=current_bulk_data,
    new_data_list=new_solutions
)
\end{lstlisting}

\paragraph{get\_bulk\_data\_arrays()}\leavevmode
\begin{lstlisting}[language=Python, caption=Get Data Arrays Method]
def get_bulk_data_arrays(self, bulk_data_list: List[BulkData]) -> List[np.ndarray]
\end{lstlisting}

\textbf{Parameters:}
\begin{itemize}
    \item \texttt{bulk\_data\_list}: List of BulkData objects
\end{itemize}

\textbf{Returns:} \texttt{List[np.ndarray]} - Data arrays from all BulkData objects

\textbf{Usage:}
\begin{lstlisting}[language=Python, caption=Get Arrays Usage]
# Extract data arrays for external processing
data_arrays = lean_manager.get_bulk_data_arrays(bulk_data_list)
for i, array in enumerate(data_arrays):
    print(f"Domain {i} data shape: {array.shape}")
\end{lstlisting}

\subsection{Analysis Methods}
\label{subsec:analysis_methods}

\paragraph{compute\_total\_mass()}\leavevmode
\begin{lstlisting}[language=Python, caption=Compute Total Mass Method]
def compute_total_mass(self, bulk_data_list: List[BulkData]) -> float
\end{lstlisting}

\textbf{Parameters:}
\begin{itemize}
    \item \texttt{bulk\_data\_list}: List of BulkData instances
\end{itemize}

\textbf{Returns:} \texttt{float} - Total mass across all domains

\textbf{Computation:} Sums mass from each domain using stored mass matrices

\textbf{Usage:}
\begin{lstlisting}[language=Python, caption=Mass Computation Usage]
# Monitor mass conservation
initial_mass = lean_manager.compute_total_mass(initial_bulk_data)
current_mass = lean_manager.compute_total_mass(current_bulk_data)

mass_change = abs(current_mass - initial_mass) / initial_mass
print(f"Relative mass change: {mass_change:.6e}")
\end{lstlisting}




\paragraph{compute\_mass\_conservation()}\leavevmode
\begin{lstlisting}[language=Python, caption=Mass Conservation Method]
def compute_mass_conservation(self, bulk_data_list: List[BulkData]) -> float
\end{lstlisting}

\textbf{Parameters:}
\begin{itemize}
    \item \texttt{bulk\_data\_list}: List of BulkData instances
\end{itemize}

\textbf{Returns:} \texttt{float} - Total mass (alias for \texttt{compute\_total\_mass})

\textbf{Note:} This method is an alias for consistency with other interfaces

\begin{warningbox}
	Both methods compute the sum of the masses of the different fields. In our framework we should compute the individual mass of the individual fields.
\end{warningbox}

\subsection{Utility Methods}
\label{subsec:utility_methods}

\paragraph{get\_num\_domains()}\leavevmode
\begin{lstlisting}[language=Python, caption=Get Number of Domains Method]
def get_num_domains(self) -> int
\end{lstlisting}

\textbf{Returns:} \texttt{int} - Number of domains managed

\textbf{Usage:}
\begin{lstlisting}[language=Python, caption=Get Domains Count Usage]
n_domains = lean_manager.get_num_domains()
print(f"Managing {n_domains} domains")
\end{lstlisting}

\paragraph{get\_domain\_info()}\leavevmode
\begin{lstlisting}[language=Python, caption=Get Domain Info Method]
def get_domain_info(self, domain_idx: int) -> DomainData
\end{lstlisting}

\textbf{Parameters:}
\begin{itemize}
    \item \texttt{domain\_idx}: Domain index
\end{itemize}

\textbf{Returns:} \texttt{DomainData} - Domain data object for inspection

\textbf{Raises:} \texttt{IndexError} for invalid domain index

\textbf{Usage:}
\begin{lstlisting}[language=Python, caption=Domain Info Usage]
# Inspect domain properties
domain_info = lean_manager.get_domain_info(0)
print(f"Domain 0: {domain_info.neq} equations, {domain_info.n_elements} elements")
print(f"Element length: {domain_info.element_length}")
\end{lstlisting}

\subsection{Testing and Validation}
\label{subsec:testing_validation}

\paragraph{test()}\leavevmode
\begin{lstlisting}[language=Python, caption=Test Method]
def test(self, 
         problems: List = None,
         discretizations: List = None,
         static_condensations: List = None) -> bool
\end{lstlisting}

\textbf{Parameters:}
\begin{itemize}
    \item \texttt{problems}: List of Problem objects for testing (optional)
    \item \texttt{discretizations}: List of discretization objects for testing (optional)
    \item \texttt{static\_condensations}: List of static condensation objects for testing (optional)
\end{itemize}

\textbf{Returns:} \texttt{bool} - True if all tests pass, False otherwise

\textbf{Test Suite:}
\begin{enumerate}
    \item \textbf{Framework Object Validation Test}: Validates provided framework objects
    \item \textbf{Domain Data Structure Test}: Validates stored domain data integrity
    \item \textbf{BulkData Creation Test}: Tests creation of primal and dual BulkData
    \item \textbf{Initialization Test}: Tests bulk data initialization
    \item \textbf{Forcing Term Computation Test}: Tests forcing term calculations
    \item \textbf{Mass Computation Test}: Tests mass conservation calculations
    \item \textbf{Bounds Checking Test}: Tests error handling for invalid indices
    \item \textbf{Utility Methods Test}: Tests helper methods
    \item \textbf{Parameter Mismatch Test}: Tests validation error detection
\end{enumerate}

\textbf{Usage:}
\begin{lstlisting}[language=Python, caption=Test Method Usage]
# Comprehensive testing with framework objects
if lean_manager.test(
    problems=problems,
    discretizations=discretizations,
    static_condensations=static_condensations
):
    print("✓ Lean BulkDataManager is fully functional")
else:
    print("✗ Issues detected in Lean BulkDataManager")

# Minimal testing without framework objects
if lean_manager.test():
    print("✓ Basic structure validation passed")
\end{lstlisting}

\textbf{Sample Test Output:}
\begin{lstlisting}[language=Python, caption=Sample Test Output]
Testing Lean BulkDataManager with 3 domains
PASS: Framework object validation passed
PASS: All domain data validated
PASS: BulkData creation tests passed
PASS: Initialization tests passed
PASS: Forcing term computation tests passed
PASS: Mass computation test passed (total_mass=1.234567e+00)
PASS: ValueError raised for negative domain index
PASS: get_num_domains() returned correct value
PASS: get_domain_info() test passed
PASS: Correctly detected wrong number of problems
PASS: Correctly detected incompatible problem neq
PASS: Parameter mismatch detection tests passed
✓ All Lean BulkDataManager tests passed!
\end{lstlisting}

\subsection{Special Methods}
\label{subsec:special_methods}

\paragraph{\_\_str\_\_()}\leavevmode
\begin{lstlisting}[language=Python, caption=String Representation Method]
def __str__(self) -> str
\end{lstlisting}

\textbf{Returns:} \texttt{str} - Human-readable summary

\textbf{Format:} \texttt{"LeanBulkDataManager(domains=N, total\_elements=M, total\_equations=K)"}

\paragraph{\_\_repr\_\_()}\leavevmode
\begin{lstlisting}[language=Python, caption=Repr Method]
def __repr__(self) -> str
\end{lstlisting}

\textbf{Returns:} \texttt{str} - Developer-oriented representation

\textbf{Format:} \texttt{"LeanBulkDataManager(n\_domains=N, domain\_elements=[...], domain\_equations=[...])"}

\textbf{Usage:}
\begin{lstlisting}[language=Python, caption=String Methods Usage]
print(str(lean_manager))
# Output: LeanBulkDataManager(domains=3, total_elements=60, total_equations=6)

print(repr(lean_manager))
# Output: LeanBulkDataManager(n_domains=3, domain_elements=[20, 20, 20], domain_equations=[2, 2, 2])
\end{lstlisting}

\subsection{Complete Usage Examples}
\label{subsec:complete_usage_examples}

\subsubsection{Standard Workflow Example}

\begin{lstlisting}[language=Python, caption=Complete Lean Manager Workflow]
from bionetflux.core.lean_bulk_data_manager import BulkDataManager
from bionetflux.core.problem import Problem
from bionetflux.core.discretization import Discretization



# Step 1: Create framework objects (problems, discretizations, static_condensations)
# problems = [...]  # List of Problem instances
# discretizations = [...]  # List of Discretization instances  
# static_condensations = [...]  # List of static condensation instances
from setup_solver import quick_setup
filename = "bionetflux.problems.reduced_ooc_problem" 
setup = quick_setup(filename, validate=True)
problems = setup.problems
discretizations = setup.global_discretization.spatial_discretizations
static_condensations = setup.static_condensations

# Step 2: Extract essential data once (memory-efficient)
domain_data_list = BulkDataManager.extract_domain_data_list(
problems=problems,
discretizations=discretizations,
static_condensations=static_condensations
)

# Step 3: Create lean manager with extracted data only
lean_manager = BulkDataManager(domain_data_list)

# Step 4: Validate compatibility (optional but recommended)
if not lean_manager.test(problems, discretizations, static_condensations):
raise RuntimeError("Framework objects incompatible with extracted data")

# Step 5: Initialize bulk data for all domains
bulk_data_list = lean_manager.initialize_all_bulk_data(
problems=problems,
discretizations=discretizations,
time=0.0
)

# Step 6: Time evolution loop
dt = 0.01
for time_step in range(1):
current_time = time_step * dt

# Compute forcing terms for implicit Euler
forcing_terms = lean_manager.compute_forcing_terms(
bulk_data_list=bulk_data_list,
problems=problems,
discretizations=discretizations,
time=current_time,
dt=dt
)

# Initialize new primal bulk object with random entries

new_bulk_data_list = []
for i, bulk_data in enumerate(bulk_data_list):
new_bulk_primal = lean_manager.create_bulk_data(
domain_index=i,
problem=problems[i],
discretization=discretizations[i],
dual=False
)

# Set random data for the primal bulk object
domain_data = lean_manager.get_domain_info(i)
random_shape = (2 * domain_data.neq, domain_data.n_elements)
random_data = np.random.rand(*random_shape) * 0.1  # Small random values
new_bulk_primal.set_data(random_data)

new_bulk_data_list.append(new_bulk_primal)

# Create new_solutions list (placeholder for actual solver results)
new_data = [bulk_data.get_data() for bulk_data in new_bulk_data_list]

# Update bulk data with new solutions
lean_manager.update_bulk_data(bulk_data_list, new_data)

# Monitor mass conservation
current_mass = lean_manager.compute_total_mass(bulk_data_list)
if time_step % 10 == 0:
print(f"Time {current_time:.3f}: Mass = {current_mass:.6e}")

print("✓ Time evolution completed with lean manager")
\end{lstlisting}



\subsubsection{Multi-Manager Example}

\begin{lstlisting}[language=Python, caption=Multiple Lean Managers from Same Data]
# Extract domain data once
domain_data_list = BulkDataManager.extract_domain_data_list(
    problems, discretizations, static_condensations
)

# Create multiple lean managers for different purposes
# (all sharing the same extracted data - no additional memory cost)

# Manager for time evolution
evolution_manager = BulkDataManager(domain_data_list)

# Manager for forcing term computation
forcing_manager = BulkDataManager(domain_data_list)

# Manager for mass conservation tracking  
conservation_manager = BulkDataManager(domain_data_list)

# Each manager can operate independently but uses same base data
initial_bulk = evolution_manager.initialize_all_bulk_data(problems, discretizations)
source_terms = forcing_manager.compute_source_terms(problems, discretizations, time=0.0)
total_mass = conservation_manager.compute_total_mass(initial_bulk)

print(f"Created 3 independent managers sharing {len(domain_data_list)} domain data objects")
\end{lstlisting}

\subsection{Method Summary Table}
\label{subsec:lean_method_summary}

\begin{longtable}{|p{5.3cm}|p{3.2cm}|p{5cm}|}
\hline
\textbf{Method} & \textbf{Returns} & \textbf{Purpose} \\
\hline
\endhead

\texttt{\_\_init\_\_} & \texttt{None} & Initialize with extracted domain data only \\
\hline

\texttt{extract\_domain\_data\_list} & \texttt{List[DomainData]} & Static factory for one-time data extraction \\
\hline

\texttt{create\_bulk\_data} & \texttt{BulkData} & Create BulkData using external framework objects \\
\hline

\texttt{initialize\_all\_bulk\_data} & \texttt{List[BulkData]} & Initialize all domains with initial conditions \\
\hline

\texttt{compute\_source\_terms} & \texttt{List[BulkData]} & Compute source terms using dual formulation \\
\hline

\texttt{compute\_forcing\_terms} & \texttt{List[np.ndarray]} & Compute forcing terms for implicit Euler \\
\hline

\texttt{update\_bulk\_data} & \texttt{None} & Update BulkData objects with new solutions \\
\hline

\texttt{compute\_total\_mass} & \texttt{float} & Calculate total mass for conservation \\
\hline

\texttt{get\_bulk\_data\_arrays} & \texttt{List[np.ndarray]} & Extract data arrays from BulkData objects \\
\hline

\texttt{get\_num\_domains} & \texttt{int} & Get number of managed domains \\
\hline

\texttt{get\_domain\_info} & \texttt{DomainData} & Access domain data for inspection \\
\hline

\texttt{test} & \texttt{bool} & Comprehensive validation and testing \\
\hline

\texttt{\_validate\_framework\_objects} & \texttt{None} & Validate framework object compatibility \\
\hline

\end{longtable}

This documentation provides an exact reference for the lean BulkDataManager class, emphasizing its memory-efficient design and parameter-based approach to framework object usage. The lean architecture minimizes memory overhead while maintaining full functionality through external object validation and flexible method interfaces.

% End of lean bulk data manager module API documentation

\begin{todobox}
	\begin{itemize}
		\item Rename module file as {\tt bulk\_data\_manager} or possibly also {\tt data\_manager}
		\item Evaluate moving (or duplicating) {\tt extract\_domain\_data\_list} to {\tt DomainData}
		\item Fix the mass computation and mass monitoring so that they compute the mass of individual fields
		\item The examples should use a toy quick-setup, now they use an ooc type problem
	\end{itemize}
\end{todobox}




\section{Lean Bulk Data Manager Module API Reference}
\label{sec:lean_bulk_data_manager_api}

This section provides an exact reference for the BulkDataManager class  based on detailed analysis of  the  \texttt{ooc1d.core.lean\_bulk\_data\_manager.BulkDataManager} module. This is an ultra-lean coordinator that minimizes memory usage by storing only essential domain data and accepting framework objects as method parameters.

\subsection{Module Overview}

The lean bulk data manager provides memory-efficient coordination for bulk operations by:
\begin{itemize}
    \item Storing only extracted essential domain data
    \item Accepting framework objects as method parameters
    \item Validating framework object compatibility
    \item Providing flexible bulk data operations without memory overhead
\end{itemize}

\subsection{Module Imports and Dependencies}

\begin{lstlisting}[language=Python, caption=Module Dependencies]
import numpy as np
from typing import List, Optional, Callable
from ooc1d.core.bulk_data import BulkData
from ooc1d.core.domain_data import DomainData
\end{lstlisting}

\subsection{BulkDataManager Class Definition}
\label{subsec:lean_bulk_data_manager_class}

\begin{lstlisting}[language=Python, caption=Class Declaration]
class BulkDataManager:
    """
    Ultra-lean coordinator for bulk operations in HDG method.
    
    This class stores only essential extracted domain data and accepts
    framework objects as parameters to methods that need them. This approach
    minimizes memory usage and increases flexibility.
    """
\end{lstlisting}

\subsection{Constructor}
\label{subsec:lean_constructor}

\paragraph{\_\_init\_\_()}\leavevmode
\begin{lstlisting}[language=Python, caption=Lean BulkDataManager Constructor]
def __init__(self, domain_data_list: List[DomainData])
\end{lstlisting}

\textbf{Parameters:}
\begin{itemize}
    \item \texttt{domain\_data\_list}: List of DomainData objects with essential extracted information
\end{itemize}

\textbf{Side Effects:} Sets \texttt{self.domain\_data\_list} attribute

\textbf{Usage:}
\begin{lstlisting}[language=Python, caption=Constructor Usage]
# Extract domain data first (see static factory method)
domain_data_list = BulkDataManager.extract_domain_data_list(
    problems, discretizations, static_condensations
)

# Create lean manager with extracted data only
lean_manager = BulkDataManager(domain_data_list)
\end{lstlisting}

\subsection{Core Attributes}
\label{subsec:lean_attributes}

\begin{longtable}{|p{3.5cm}|p{3.5cm}|p{6cm}|}
\hline
\textbf{Attribute} & \textbf{Type} & \textbf{Description} \\
\hline
\endhead

\texttt{domain\_data\_list} & \texttt{List[DomainData]} & List of essential domain data objects (only stored attribute) \\
\hline

\end{longtable}

\textbf{Memory Efficiency:} The lean manager stores \textbf{only} the domain data list, avoiding storage of large framework objects.

\subsection{Validation Methods}
\label{subsec:validation_methods}

\paragraph{\_validate\_framework\_objects()}\leavevmode
\begin{lstlisting}[language=Python, caption=Framework Validation Method]
def _validate_framework_objects(self, 
                               problems: List = None,
                               discretizations: List = None, 
                               static_condensations: List = None,
                               operation_name: str = "operation") -> None
\end{lstlisting}

\textbf{Parameters:}
\begin{itemize}
    \item \texttt{problems}: List of Problem objects to validate (optional)
    \item \texttt{discretizations}: List of discretization objects to validate (optional)
    \item \texttt{static\_condensations}: List of static condensation objects to validate (optional)
    \item \texttt{operation\_name}: Name of operation for error messages (default: ``operation")
\end{itemize}

\textbf{Returns:} \texttt{None}

\textbf{Raises:} \texttt{ValueError} for incompatible framework objects

\textbf{Validation Checks:}
\begin{enumerate}
    \item \textbf{List Length Validation}: All provided lists must match domain count
    \item \textbf{Problem Validation}: \texttt{neq} attribute must match stored domain data
    \item \textbf{Discretization Validation}: \texttt{n\_elements}, nodes, and \texttt{element\_length} compatibility
    \item \textbf{Static Condensation Validation}: Matrix compatibility and method availability
\end{enumerate}

\textbf{Usage:}
\begin{lstlisting}[language=Python, caption=Validation Usage]
try:
    lean_manager._validate_framework_objects(
        problems=problems,
        discretizations=discretizations,
        operation_name="my_operation"
    )
    print("✓ Framework objects are compatible")
except ValueError as e:
    print(f"✗ Validation failed: {e}")
\end{lstlisting}

\subsection{Static Factory Methods}
\label{subsec:static_factory_methods}

\paragraph{extract\_domain\_data\_list()}\leavevmode
\begin{lstlisting}[language=Python, caption=Domain Data Extraction Method]
@staticmethod
def extract_domain_data_list(problems: List, 
                            discretizations: List, 
                            static_condensations: List) -> List[DomainData]
\end{lstlisting}

\textbf{Parameters:}
\begin{itemize}
    \item \texttt{problems}: List of Problem instances
    \item \texttt{discretizations}: List of discretization instances
    \item \texttt{static\_condensations}: List of static condensation instances
\end{itemize}

\textbf{Returns:} \texttt{List[DomainData]} - Extracted essential domain information

\textbf{Purpose:} Static factory method to extract and store essential data once for reuse

\textbf{Usage:}
\begin{lstlisting}[language=Python, caption=Domain Data Extraction Usage]
# One-time extraction of essential domain data
domain_data_list = BulkDataManager.extract_domain_data_list(
    problems=problems,
    discretizations=discretizations,
    static_condensations=static_condensations
)

# Can create multiple lean managers with same extracted data
lean_manager_1 = BulkDataManager(domain_data_list)
lean_manager_2 = BulkDataManager(domain_data_list)
\end{lstlisting}

\paragraph{\_extract\_single\_domain\_data()}\leavevmode
\begin{lstlisting}[language=Python, caption=Single Domain Extraction Method]
@staticmethod
def _extract_single_domain_data(problem, discretization, sc, domain_idx: int) -> DomainData
\end{lstlisting}

\textbf{Parameters:}
\begin{itemize}
    \item \texttt{problem}: Problem object for single domain
    \item \texttt{discretization}: Discretization object for single domain
    \item \texttt{sc}: Static condensation object for single domain
    \item \texttt{domain\_idx}: Domain index for error reporting
\end{itemize}

\textbf{Returns:} \texttt{DomainData} - Extracted essential data for single domain

\textbf{Extraction Process:}
\begin{enumerate}
    \item Extract matrices from static condensation (\texttt{M}, \texttt{T})
    \item Extract initial conditions with multiple access pattern support
    \item Extract forcing functions with multiple access pattern support
    \item Create DomainData object with essential information
\end{enumerate}

\subsection{BulkData Creation Methods}
\label{subsec_bulk_data_creation}

Bulk data can be dual ({\tt dual = True}) or primal ({\tt dual = False}). When representing a physical object $w$ as a primal bulk data object, the two data points $w_0$ and $w_1$ relative to the element $E$ are the values of the degrees of freedom such that on $E$ we have \(w = \sum_{i=0}^1 w_i e_i\), where $e_0$, $e_1$ are the chosen basis for the P1 space on $E$. For the nodal basis, $w_0$ and $w_1$ are the values of $w$ at the two vertices of $E$. When representing $w$ with a dual bulk object, $w_0$ and $w_1$ are the actions of $w$ on the basis functions: \(w_i = \int_E w e_i\). The choice on whether to use a primal or a dual bulk data object to represent a physical function, depends on its role in the problem/implementation. The solution will typically be represented by a primal bulk data object, while the right hand side will be represented by a dual object. The {\tt BulkData} creation methods handles these possibilities. 


\paragraph{create\_bulk\_data()}\leavevmode
\begin{lstlisting}[language=Python, caption=Create BulkData Method]
def create_bulk_data(self, 
                    domain_index: int, 
                    problem, 
                    discretization, 
                    dual: bool = False) -> BulkData
\end{lstlisting}

\textbf{Parameters:}
\begin{itemize}
    \item \texttt{domain\_index}: Index of the domain (0 to \texttt{n\_domains-1})
    \item \texttt{problem}: Problem object for this domain
    \item \texttt{discretization}: Discretization object for this domain
    \item \texttt{dual}: Whether to use dual formulation (default: False)
\end{itemize}

\textbf{Returns:} \texttt{BulkData} - New BulkData object

\textbf{Raises:} \texttt{ValueError} for invalid domain index

\textbf{Usage:}
\begin{lstlisting}[language=Python, caption=BulkData Creation Usage]
# Create primal BulkData for domain 0
bulk_data_primal = lean_manager.create_bulk_data(
    domain_index=0,
    problem=problems[0],
    discretization=discretizations[0],
    dual=False
)

# Create dual BulkData for forcing terms
bulk_data_dual = lean_manager.create_bulk_data(
    domain_index=0,
    problem=problems[0],
    discretization=discretizations[0],
    dual=True
)
\end{lstlisting}

\subsection{Bulk Operations Methods}
\label{subsec:bulk_operations}

\paragraph{compute\_source\_terms()}\leavevmode
\begin{lstlisting}[language=Python, caption=Compute Source Terms Method]
def compute_source_terms(self,
                        problems: List,
                        discretizations: List,
                        time: float) -> List[BulkData]
\end{lstlisting}


\textbf{Parameters:}
\begin{itemize}
    \item \texttt{problems}: List of Problem objects
    \item \texttt{discretizations}: List of discretization objects
    \item \texttt{time}: Current time for evaluation
\end{itemize}

\textbf{Returns:} \texttt{List[BulkData]} - Source terms for all domains using dual formulation

\textbf{Process:}
\begin{enumerate}
    \item Validate framework objects against stored domain data
    \item Create dual BulkData for each domain
    \item Set forcing functions from problems at specified time
    \item Return list of integrated source terms
\end{enumerate}

\textbf{Usage:}
\begin{lstlisting}[language=Python, caption=Source Terms Usage]
# Compute source terms at time t=0.5
source_terms = lean_manager.compute_source_terms(
    problems=problems,
    discretizations=discretizations,
    time=0.5
)

print(f"Computed {len(source_terms)} source term BulkData objects")
\end{lstlisting}

\paragraph{compute\_forcing\_terms()}\leavevmode
\begin{lstlisting}[language=Python, caption=Compute Forcing Terms Method]
def compute_forcing_terms(self, 
                          bulk_data_list: List[BulkData],
                          problems: List,
                          discretizations: List, 
                          time: float, 
                          dt: float) -> List[np.ndarray]
\end{lstlisting}

\textbf{Parameters:}
\begin{itemize}
    \item \texttt{bulk\_data\_list}: List of current BulkData solutions
    \item \texttt{problems}: List of Problem objects
    \item \texttt{discretizations}: List of discretization objects
    \item \texttt{time}: Current time
    \item \texttt{dt}: Time step size
\end{itemize}

\textbf{Returns:} \texttt{List[np.ndarray]} - Forcing term arrays (dual) for implicit Euler

\textbf{Computation:} For each domain: \texttt{forcing\_term = dt * force\_contrib + M * U\_old}

\textbf{Usage:}
\begin{lstlisting}[language=Python, caption=Forcing Terms Usage]
# Compute forcing terms for implicit Euler step
current_bulk_data = [...]  # Current solution
forcing_terms = lean_manager.compute_forcing_terms(
    bulk_data_list=current_bulk_data,
    problems=problems,
    discretizations=discretizations,
    time=0.1,
    dt=0.01
)

# forcing_terms[i] has shape (2*neq, n_elements) for domain i
\end{lstlisting}

\subsection{Initialization Methods}
\label{subsec:initialization_methods}

\paragraph{initialize\_all\_bulk\_data()}\leavevmode
\begin{lstlisting}[language=Python, caption=Initialize All BulkData Method]
def initialize_all_bulk_data(self, 
                            problems: List,
                            discretizations: List,
                            time: float = 0.0) -> List[BulkData]
\end{lstlisting}

\textbf{Parameters:}
\begin{itemize}
    \item \texttt{problems}: List of Problem objects
    \item \texttt{discretizations}: List of discretization objects
    \item \texttt{time}: Initial time (default: 0.0)
\end{itemize}

\textbf{Returns:} \texttt{List[BulkData]} - Initialized BulkData objects containing primal representation of initial conditions for all domains; calls {\tt initialize\_bulk\_data\_from\_initial\_conditions} for all domains

\textbf{Usage:}
\begin{lstlisting}[language=Python, caption=Initialize All Usage]
# Initialize all domains with initial conditions
initial_bulk_data = lean_manager.initialize_all_bulk_data(
    problems=problems,
    discretizations=discretizations,
    time=0.0
)

print(f"Initialized {len(initial_bulk_data)} BulkData objects")
\end{lstlisting}

\paragraph{initialize\_bulk\_data\_from\_initial\_conditions()}\leavevmode
\begin{lstlisting}[language=Python, caption=Initialize Single Domain Method]
def initialize_bulk_data_from_initial_conditions(self, 
                                                domain_index: int,
                                                problem,
                                                discretization,
                                                time: float = 0.0) -> BulkData
\end{lstlisting}

\textbf{Parameters:}
\begin{itemize}
    \item \texttt{domain\_index}: Index of domain to initialize
    \item \texttt{problem}: Problem object for this domain
    \item \texttt{discretization}: Discretization object for this domain
    \item \texttt{time}: Initial time (default: 0.0)
\end{itemize}

\textbf{Returns:} \texttt{BulkData} - Initialized BulkData object containing primal representation of initial conditions

\textbf{Process:}
\begin{enumerate}
    \item Validate domain index and framework objects
    \item Create BulkData with primal formulation
    \item Set initial conditions from stored domain data
    \item Default to zero if no initial conditions available
\end{enumerate}

\textbf{Usage:}
\begin{lstlisting}[language=Python, caption=Initialize Single Domain Usage]
# Initialize specific domain
domain_0_bulk = lean_manager.initialize_bulk_data_from_initial_conditions(
    domain_index=0,
    problem=problems[0],
    discretization=discretizations[0],
    time=0.0
)
\end{lstlisting}

\subsection{Data Management Methods}
\label{subsec:data_management}

\paragraph{update\_bulk\_data()}\leavevmode
\begin{lstlisting}[language=Python, caption=Update BulkData Method]
def update_bulk_data(self, bulk_data_list: List[BulkData], new_data_list: List[np.ndarray])
\end{lstlisting}

\textbf{Parameters:}
\begin{itemize}
    \item \texttt{bulk\_data\_list}: List of BulkData objects to update
    \item \texttt{new\_data\_list}: List of new bulk solution arrays
\end{itemize}

\textbf{Returns:} \texttt{None}

\textbf{Side Effects:} Updates data in all BulkData objects

\textbf{Validation:}
\begin{itemize}
    \item Lists must have matching lengths
    \item New data must have compatible shapes
    \item New data must not contain NaN or infinite values
\end{itemize}

\textbf{Usage:}
\begin{lstlisting}[language=Python, caption=Update BulkData Usage]
# Update bulk data with new solution
new_solutions = [...]  # List of numpy arrays
lean_manager.update_bulk_data(
    bulk_data_list=current_bulk_data,
    new_data_list=new_solutions
)
\end{lstlisting}

\paragraph{get\_bulk\_data\_arrays()}\leavevmode
\begin{lstlisting}[language=Python, caption=Get Data Arrays Method]
def get_bulk_data_arrays(self, bulk_data_list: List[BulkData]) -> List[np.ndarray]
\end{lstlisting}

\textbf{Parameters:}
\begin{itemize}
    \item \texttt{bulk\_data\_list}: List of BulkData objects
\end{itemize}

\textbf{Returns:} \texttt{List[np.ndarray]} - Data arrays from all BulkData objects

\textbf{Usage:}
\begin{lstlisting}[language=Python, caption=Get Arrays Usage]
# Extract data arrays for external processing
data_arrays = lean_manager.get_bulk_data_arrays(bulk_data_list)
for i, array in enumerate(data_arrays):
    print(f"Domain {i} data shape: {array.shape}")
\end{lstlisting}

\subsection{Analysis Methods}
\label{subsec:analysis_methods}

\paragraph{compute\_total\_mass()}\leavevmode
\begin{lstlisting}[language=Python, caption=Compute Total Mass Method]
def compute_total_mass(self, bulk_data_list: List[BulkData]) -> float
\end{lstlisting}

\textbf{Parameters:}
\begin{itemize}
    \item \texttt{bulk\_data\_list}: List of BulkData instances
\end{itemize}

\textbf{Returns:} \texttt{float} - Total mass across all domains

\textbf{Computation:} Sums mass from each domain using stored mass matrices

\textbf{Usage:}
\begin{lstlisting}[language=Python, caption=Mass Computation Usage]
# Monitor mass conservation
initial_mass = lean_manager.compute_total_mass(initial_bulk_data)
current_mass = lean_manager.compute_total_mass(current_bulk_data)

mass_change = abs(current_mass - initial_mass) / initial_mass
print(f"Relative mass change: {mass_change:.6e}")
\end{lstlisting}




\paragraph{compute\_mass\_conservation()}\leavevmode
\begin{lstlisting}[language=Python, caption=Mass Conservation Method]
def compute_mass_conservation(self, bulk_data_list: List[BulkData]) -> float
\end{lstlisting}

\textbf{Parameters:}
\begin{itemize}
    \item \texttt{bulk\_data\_list}: List of BulkData instances
\end{itemize}

\textbf{Returns:} \texttt{float} - Total mass (alias for \texttt{compute\_total\_mass})

\textbf{Note:} This method is an alias for consistency with other interfaces

\begin{warningbox}
	Both methods compute the sum of the masses of the different fields. In our framework we should compute the individual mass of the individual fields.
\end{warningbox}

\subsection{Utility Methods}
\label{subsec:utility_methods}

\paragraph{get\_num\_domains()}\leavevmode
\begin{lstlisting}[language=Python, caption=Get Number of Domains Method]
def get_num_domains(self) -> int
\end{lstlisting}

\textbf{Returns:} \texttt{int} - Number of domains managed

\textbf{Usage:}
\begin{lstlisting}[language=Python, caption=Get Domains Count Usage]
n_domains = lean_manager.get_num_domains()
print(f"Managing {n_domains} domains")
\end{lstlisting}

\paragraph{get\_domain\_info()}\leavevmode
\begin{lstlisting}[language=Python, caption=Get Domain Info Method]
def get_domain_info(self, domain_idx: int) -> DomainData
\end{lstlisting}

\textbf{Parameters:}
\begin{itemize}
    \item \texttt{domain\_idx}: Domain index
\end{itemize}

\textbf{Returns:} \texttt{DomainData} - Domain data object for inspection

\textbf{Raises:} \texttt{IndexError} for invalid domain index

\textbf{Usage:}
\begin{lstlisting}[language=Python, caption=Domain Info Usage]
# Inspect domain properties
domain_info = lean_manager.get_domain_info(0)
print(f"Domain 0: {domain_info.neq} equations, {domain_info.n_elements} elements")
print(f"Element length: {domain_info.element_length}")
\end{lstlisting}

\subsection{Testing and Validation}
\label{subsec:testing_validation}

\paragraph{test()}\leavevmode
\begin{lstlisting}[language=Python, caption=Test Method]
def test(self, 
         problems: List = None,
         discretizations: List = None,
         static_condensations: List = None) -> bool
\end{lstlisting}

\textbf{Parameters:}
\begin{itemize}
    \item \texttt{problems}: List of Problem objects for testing (optional)
    \item \texttt{discretizations}: List of discretization objects for testing (optional)
    \item \texttt{static\_condensations}: List of static condensation objects for testing (optional)
\end{itemize}

\textbf{Returns:} \texttt{bool} - True if all tests pass, False otherwise

\textbf{Test Suite:}
\begin{enumerate}
    \item \textbf{Framework Object Validation Test}: Validates provided framework objects
    \item \textbf{Domain Data Structure Test}: Validates stored domain data integrity
    \item \textbf{BulkData Creation Test}: Tests creation of primal and dual BulkData
    \item \textbf{Initialization Test}: Tests bulk data initialization
    \item \textbf{Forcing Term Computation Test}: Tests forcing term calculations
    \item \textbf{Mass Computation Test}: Tests mass conservation calculations
    \item \textbf{Bounds Checking Test}: Tests error handling for invalid indices
    \item \textbf{Utility Methods Test}: Tests helper methods
    \item \textbf{Parameter Mismatch Test}: Tests validation error detection
\end{enumerate}

\textbf{Usage:}
\begin{lstlisting}[language=Python, caption=Test Method Usage]
# Comprehensive testing with framework objects
if lean_manager.test(
    problems=problems,
    discretizations=discretizations,
    static_condensations=static_condensations
):
    print("✓ Lean BulkDataManager is fully functional")
else:
    print("✗ Issues detected in Lean BulkDataManager")

# Minimal testing without framework objects
if lean_manager.test():
    print("✓ Basic structure validation passed")
\end{lstlisting}

\textbf{Sample Test Output:}
\begin{lstlisting}[language=Python, caption=Sample Test Output]
Testing Lean BulkDataManager with 3 domains
PASS: Framework object validation passed
PASS: All domain data validated
PASS: BulkData creation tests passed
PASS: Initialization tests passed
PASS: Forcing term computation tests passed
PASS: Mass computation test passed (total_mass=1.234567e+00)
PASS: ValueError raised for negative domain index
PASS: get_num_domains() returned correct value
PASS: get_domain_info() test passed
PASS: Correctly detected wrong number of problems
PASS: Correctly detected incompatible problem neq
PASS: Parameter mismatch detection tests passed
✓ All Lean BulkDataManager tests passed!
\end{lstlisting}

\subsection{Special Methods}
\label{subsec:special_methods}

\paragraph{\_\_str\_\_()}\leavevmode
\begin{lstlisting}[language=Python, caption=String Representation Method]
def __str__(self) -> str
\end{lstlisting}

\textbf{Returns:} \texttt{str} - Human-readable summary

\textbf{Format:} \texttt{"LeanBulkDataManager(domains=N, total\_elements=M, total\_equations=K)"}

\paragraph{\_\_repr\_\_()}\leavevmode
\begin{lstlisting}[language=Python, caption=Repr Method]
def __repr__(self) -> str
\end{lstlisting}

\textbf{Returns:} \texttt{str} - Developer-oriented representation

\textbf{Format:} \texttt{"LeanBulkDataManager(n\_domains=N, domain\_elements=[...], domain\_equations=[...])"}

\textbf{Usage:}
\begin{lstlisting}[language=Python, caption=String Methods Usage]
print(str(lean_manager))
# Output: LeanBulkDataManager(domains=3, total_elements=60, total_equations=6)

print(repr(lean_manager))
# Output: LeanBulkDataManager(n_domains=3, domain_elements=[20, 20, 20], domain_equations=[2, 2, 2])
\end{lstlisting}

\subsection{Complete Usage Examples}
\label{subsec:complete_usage_examples}

\subsubsection{Standard Workflow Example}

\begin{lstlisting}[language=Python, caption=Complete Lean Manager Workflow]
from bionetflux.core.lean_bulk_data_manager import BulkDataManager
from bionetflux.core.problem import Problem
from bionetflux.core.discretization import Discretization



# Step 1: Create framework objects (problems, discretizations, static_condensations)
# problems = [...]  # List of Problem instances
# discretizations = [...]  # List of Discretization instances  
# static_condensations = [...]  # List of static condensation instances
from setup_solver import quick_setup
filename = "bionetflux.problems.reduced_ooc_problem" 
setup = quick_setup(filename, validate=True)
problems = setup.problems
discretizations = setup.global_discretization.spatial_discretizations
static_condensations = setup.static_condensations

# Step 2: Extract essential data once (memory-efficient)
domain_data_list = BulkDataManager.extract_domain_data_list(
problems=problems,
discretizations=discretizations,
static_condensations=static_condensations
)

# Step 3: Create lean manager with extracted data only
lean_manager = BulkDataManager(domain_data_list)

# Step 4: Validate compatibility (optional but recommended)
if not lean_manager.test(problems, discretizations, static_condensations):
raise RuntimeError("Framework objects incompatible with extracted data")

# Step 5: Initialize bulk data for all domains
bulk_data_list = lean_manager.initialize_all_bulk_data(
problems=problems,
discretizations=discretizations,
time=0.0
)

# Step 6: Time evolution loop
dt = 0.01
for time_step in range(1):
current_time = time_step * dt

# Compute forcing terms for implicit Euler
forcing_terms = lean_manager.compute_forcing_terms(
bulk_data_list=bulk_data_list,
problems=problems,
discretizations=discretizations,
time=current_time,
dt=dt
)

# Initialize new primal bulk object with random entries

new_bulk_data_list = []
for i, bulk_data in enumerate(bulk_data_list):
new_bulk_primal = lean_manager.create_bulk_data(
domain_index=i,
problem=problems[i],
discretization=discretizations[i],
dual=False
)

# Set random data for the primal bulk object
domain_data = lean_manager.get_domain_info(i)
random_shape = (2 * domain_data.neq, domain_data.n_elements)
random_data = np.random.rand(*random_shape) * 0.1  # Small random values
new_bulk_primal.set_data(random_data)

new_bulk_data_list.append(new_bulk_primal)

# Create new_solutions list (placeholder for actual solver results)
new_data = [bulk_data.get_data() for bulk_data in new_bulk_data_list]

# Update bulk data with new solutions
lean_manager.update_bulk_data(bulk_data_list, new_data)

# Monitor mass conservation
current_mass = lean_manager.compute_total_mass(bulk_data_list)
if time_step % 10 == 0:
print(f"Time {current_time:.3f}: Mass = {current_mass:.6e}")

print("✓ Time evolution completed with lean manager")
\end{lstlisting}



\subsubsection{Multi-Manager Example}

\begin{lstlisting}[language=Python, caption=Multiple Lean Managers from Same Data]
# Extract domain data once
domain_data_list = BulkDataManager.extract_domain_data_list(
    problems, discretizations, static_condensations
)

# Create multiple lean managers for different purposes
# (all sharing the same extracted data - no additional memory cost)

# Manager for time evolution
evolution_manager = BulkDataManager(domain_data_list)

# Manager for forcing term computation
forcing_manager = BulkDataManager(domain_data_list)

# Manager for mass conservation tracking  
conservation_manager = BulkDataManager(domain_data_list)

# Each manager can operate independently but uses same base data
initial_bulk = evolution_manager.initialize_all_bulk_data(problems, discretizations)
source_terms = forcing_manager.compute_source_terms(problems, discretizations, time=0.0)
total_mass = conservation_manager.compute_total_mass(initial_bulk)

print(f"Created 3 independent managers sharing {len(domain_data_list)} domain data objects")
\end{lstlisting}

\subsection{Method Summary Table}
\label{subsec:lean_method_summary}

\begin{longtable}{|p{5.3cm}|p{3.2cm}|p{5cm}|}
\hline
\textbf{Method} & \textbf{Returns} & \textbf{Purpose} \\
\hline
\endhead

\texttt{\_\_init\_\_} & \texttt{None} & Initialize with extracted domain data only \\
\hline

\texttt{extract\_domain\_data\_list} & \texttt{List[DomainData]} & Static factory for one-time data extraction \\
\hline

\texttt{create\_bulk\_data} & \texttt{BulkData} & Create BulkData using external framework objects \\
\hline

\texttt{initialize\_all\_bulk\_data} & \texttt{List[BulkData]} & Initialize all domains with initial conditions \\
\hline

\texttt{compute\_source\_terms} & \texttt{List[BulkData]} & Compute source terms using dual formulation \\
\hline

\texttt{compute\_forcing\_terms} & \texttt{List[np.ndarray]} & Compute forcing terms for implicit Euler \\
\hline

\texttt{update\_bulk\_data} & \texttt{None} & Update BulkData objects with new solutions \\
\hline

\texttt{compute\_total\_mass} & \texttt{float} & Calculate total mass for conservation \\
\hline

\texttt{get\_bulk\_data\_arrays} & \texttt{List[np.ndarray]} & Extract data arrays from BulkData objects \\
\hline

\texttt{get\_num\_domains} & \texttt{int} & Get number of managed domains \\
\hline

\texttt{get\_domain\_info} & \texttt{DomainData} & Access domain data for inspection \\
\hline

\texttt{test} & \texttt{bool} & Comprehensive validation and testing \\
\hline

\texttt{\_validate\_framework\_objects} & \texttt{None} & Validate framework object compatibility \\
\hline

\end{longtable}

This documentation provides an exact reference for the lean BulkDataManager class, emphasizing its memory-efficient design and parameter-based approach to framework object usage. The lean architecture minimizes memory overhead while maintaining full functionality through external object validation and flexible method interfaces.

% End of lean bulk data manager module API documentation

\begin{todobox}
	\begin{itemize}
		\item Rename module file as {\tt bulk\_data\_manager} or possibly also {\tt data\_manager}
		\item Evaluate moving (or duplicating) {\tt extract\_domain\_data\_list} to {\tt DomainData}
		\item Fix the mass computation and mass monitoring so that they compute the mass of individual fields
		\item The examples should use a toy quick-setup, now they use an ooc type problem
	\end{itemize}
\end{todobox}




\section{Lean Bulk Data Manager Module API Reference}
\label{sec:lean_bulk_data_manager_api}

This section provides an exact reference for the BulkDataManager class  based on detailed analysis of  the  \texttt{ooc1d.core.lean\_bulk\_data\_manager.BulkDataManager} module. This is an ultra-lean coordinator that minimizes memory usage by storing only essential domain data and accepting framework objects as method parameters.

\subsection{Module Overview}

The lean bulk data manager provides memory-efficient coordination for bulk operations by:
\begin{itemize}
    \item Storing only extracted essential domain data
    \item Accepting framework objects as method parameters
    \item Validating framework object compatibility
    \item Providing flexible bulk data operations without memory overhead
\end{itemize}

\subsection{Module Imports and Dependencies}

\begin{lstlisting}[language=Python, caption=Module Dependencies]
import numpy as np
from typing import List, Optional, Callable
from ooc1d.core.bulk_data import BulkData
from ooc1d.core.domain_data import DomainData
\end{lstlisting}

\subsection{BulkDataManager Class Definition}
\label{subsec:lean_bulk_data_manager_class}

\begin{lstlisting}[language=Python, caption=Class Declaration]
class BulkDataManager:
    """
    Ultra-lean coordinator for bulk operations in HDG method.
    
    This class stores only essential extracted domain data and accepts
    framework objects as parameters to methods that need them. This approach
    minimizes memory usage and increases flexibility.
    """
\end{lstlisting}

\subsection{Constructor}
\label{subsec:lean_constructor}

\paragraph{\_\_init\_\_()}\leavevmode
\begin{lstlisting}[language=Python, caption=Lean BulkDataManager Constructor]
def __init__(self, domain_data_list: List[DomainData])
\end{lstlisting}

\textbf{Parameters:}
\begin{itemize}
    \item \texttt{domain\_data\_list}: List of DomainData objects with essential extracted information
\end{itemize}

\textbf{Side Effects:} Sets \texttt{self.domain\_data\_list} attribute

\textbf{Usage:}
\begin{lstlisting}[language=Python, caption=Constructor Usage]
# Extract domain data first (see static factory method)
domain_data_list = BulkDataManager.extract_domain_data_list(
    problems, discretizations, static_condensations
)

# Create lean manager with extracted data only
lean_manager = BulkDataManager(domain_data_list)
\end{lstlisting}

\subsection{Core Attributes}
\label{subsec:lean_attributes}

\begin{longtable}{|p{3.5cm}|p{3.5cm}|p{6cm}|}
\hline
\textbf{Attribute} & \textbf{Type} & \textbf{Description} \\
\hline
\endhead

\texttt{domain\_data\_list} & \texttt{List[DomainData]} & List of essential domain data objects (only stored attribute) \\
\hline

\end{longtable}

\textbf{Memory Efficiency:} The lean manager stores \textbf{only} the domain data list, avoiding storage of large framework objects.

\subsection{Validation Methods}
\label{subsec:validation_methods}

\paragraph{\_validate\_framework\_objects()}\leavevmode
\begin{lstlisting}[language=Python, caption=Framework Validation Method]
def _validate_framework_objects(self, 
                               problems: List = None,
                               discretizations: List = None, 
                               static_condensations: List = None,
                               operation_name: str = "operation") -> None
\end{lstlisting}

\textbf{Parameters:}
\begin{itemize}
    \item \texttt{problems}: List of Problem objects to validate (optional)
    \item \texttt{discretizations}: List of discretization objects to validate (optional)
    \item \texttt{static\_condensations}: List of static condensation objects to validate (optional)
    \item \texttt{operation\_name}: Name of operation for error messages (default: ``operation")
\end{itemize}

\textbf{Returns:} \texttt{None}

\textbf{Raises:} \texttt{ValueError} for incompatible framework objects

\textbf{Validation Checks:}
\begin{enumerate}
    \item \textbf{List Length Validation}: All provided lists must match domain count
    \item \textbf{Problem Validation}: \texttt{neq} attribute must match stored domain data
    \item \textbf{Discretization Validation}: \texttt{n\_elements}, nodes, and \texttt{element\_length} compatibility
    \item \textbf{Static Condensation Validation}: Matrix compatibility and method availability
\end{enumerate}

\textbf{Usage:}
\begin{lstlisting}[language=Python, caption=Validation Usage]
try:
    lean_manager._validate_framework_objects(
        problems=problems,
        discretizations=discretizations,
        operation_name="my_operation"
    )
    print("✓ Framework objects are compatible")
except ValueError as e:
    print(f"✗ Validation failed: {e}")
\end{lstlisting}

\subsection{Static Factory Methods}
\label{subsec:static_factory_methods}

\paragraph{extract\_domain\_data\_list()}\leavevmode
\begin{lstlisting}[language=Python, caption=Domain Data Extraction Method]
@staticmethod
def extract_domain_data_list(problems: List, 
                            discretizations: List, 
                            static_condensations: List) -> List[DomainData]
\end{lstlisting}

\textbf{Parameters:}
\begin{itemize}
    \item \texttt{problems}: List of Problem instances
    \item \texttt{discretizations}: List of discretization instances
    \item \texttt{static\_condensations}: List of static condensation instances
\end{itemize}

\textbf{Returns:} \texttt{List[DomainData]} - Extracted essential domain information

\textbf{Purpose:} Static factory method to extract and store essential data once for reuse

\textbf{Usage:}
\begin{lstlisting}[language=Python, caption=Domain Data Extraction Usage]
# One-time extraction of essential domain data
domain_data_list = BulkDataManager.extract_domain_data_list(
    problems=problems,
    discretizations=discretizations,
    static_condensations=static_condensations
)

# Can create multiple lean managers with same extracted data
lean_manager_1 = BulkDataManager(domain_data_list)
lean_manager_2 = BulkDataManager(domain_data_list)
\end{lstlisting}

\paragraph{\_extract\_single\_domain\_data()}\leavevmode
\begin{lstlisting}[language=Python, caption=Single Domain Extraction Method]
@staticmethod
def _extract_single_domain_data(problem, discretization, sc, domain_idx: int) -> DomainData
\end{lstlisting}

\textbf{Parameters:}
\begin{itemize}
    \item \texttt{problem}: Problem object for single domain
    \item \texttt{discretization}: Discretization object for single domain
    \item \texttt{sc}: Static condensation object for single domain
    \item \texttt{domain\_idx}: Domain index for error reporting
\end{itemize}

\textbf{Returns:} \texttt{DomainData} - Extracted essential data for single domain

\textbf{Extraction Process:}
\begin{enumerate}
    \item Extract matrices from static condensation (\texttt{M}, \texttt{T})
    \item Extract initial conditions with multiple access pattern support
    \item Extract forcing functions with multiple access pattern support
    \item Create DomainData object with essential information
\end{enumerate}

\subsection{BulkData Creation Methods}
\label{subsec_bulk_data_creation}

Bulk data can be dual ({\tt dual = True}) or primal ({\tt dual = False}). When representing a physical object $w$ as a primal bulk data object, the two data points $w_0$ and $w_1$ relative to the element $E$ are the values of the degrees of freedom such that on $E$ we have \(w = \sum_{i=0}^1 w_i e_i\), where $e_0$, $e_1$ are the chosen basis for the P1 space on $E$. For the nodal basis, $w_0$ and $w_1$ are the values of $w$ at the two vertices of $E$. When representing $w$ with a dual bulk object, $w_0$ and $w_1$ are the actions of $w$ on the basis functions: \(w_i = \int_E w e_i\). The choice on whether to use a primal or a dual bulk data object to represent a physical function, depends on its role in the problem/implementation. The solution will typically be represented by a primal bulk data object, while the right hand side will be represented by a dual object. The {\tt BulkData} creation methods handles these possibilities. 


\paragraph{create\_bulk\_data()}\leavevmode
\begin{lstlisting}[language=Python, caption=Create BulkData Method]
def create_bulk_data(self, 
                    domain_index: int, 
                    problem, 
                    discretization, 
                    dual: bool = False) -> BulkData
\end{lstlisting}

\textbf{Parameters:}
\begin{itemize}
    \item \texttt{domain\_index}: Index of the domain (0 to \texttt{n\_domains-1})
    \item \texttt{problem}: Problem object for this domain
    \item \texttt{discretization}: Discretization object for this domain
    \item \texttt{dual}: Whether to use dual formulation (default: False)
\end{itemize}

\textbf{Returns:} \texttt{BulkData} - New BulkData object

\textbf{Raises:} \texttt{ValueError} for invalid domain index

\textbf{Usage:}
\begin{lstlisting}[language=Python, caption=BulkData Creation Usage]
# Create primal BulkData for domain 0
bulk_data_primal = lean_manager.create_bulk_data(
    domain_index=0,
    problem=problems[0],
    discretization=discretizations[0],
    dual=False
)

# Create dual BulkData for forcing terms
bulk_data_dual = lean_manager.create_bulk_data(
    domain_index=0,
    problem=problems[0],
    discretization=discretizations[0],
    dual=True
)
\end{lstlisting}

\subsection{Bulk Operations Methods}
\label{subsec:bulk_operations}

\paragraph{compute\_source\_terms()}\leavevmode
\begin{lstlisting}[language=Python, caption=Compute Source Terms Method]
def compute_source_terms(self,
                        problems: List,
                        discretizations: List,
                        time: float) -> List[BulkData]
\end{lstlisting}


\textbf{Parameters:}
\begin{itemize}
    \item \texttt{problems}: List of Problem objects
    \item \texttt{discretizations}: List of discretization objects
    \item \texttt{time}: Current time for evaluation
\end{itemize}

\textbf{Returns:} \texttt{List[BulkData]} - Source terms for all domains using dual formulation

\textbf{Process:}
\begin{enumerate}
    \item Validate framework objects against stored domain data
    \item Create dual BulkData for each domain
    \item Set forcing functions from problems at specified time
    \item Return list of integrated source terms
\end{enumerate}

\textbf{Usage:}
\begin{lstlisting}[language=Python, caption=Source Terms Usage]
# Compute source terms at time t=0.5
source_terms = lean_manager.compute_source_terms(
    problems=problems,
    discretizations=discretizations,
    time=0.5
)

print(f"Computed {len(source_terms)} source term BulkData objects")
\end{lstlisting}

\paragraph{compute\_forcing\_terms()}\leavevmode
\begin{lstlisting}[language=Python, caption=Compute Forcing Terms Method]
def compute_forcing_terms(self, 
                          bulk_data_list: List[BulkData],
                          problems: List,
                          discretizations: List, 
                          time: float, 
                          dt: float) -> List[np.ndarray]
\end{lstlisting}

\textbf{Parameters:}
\begin{itemize}
    \item \texttt{bulk\_data\_list}: List of current BulkData solutions
    \item \texttt{problems}: List of Problem objects
    \item \texttt{discretizations}: List of discretization objects
    \item \texttt{time}: Current time
    \item \texttt{dt}: Time step size
\end{itemize}

\textbf{Returns:} \texttt{List[np.ndarray]} - Forcing term arrays (dual) for implicit Euler

\textbf{Computation:} For each domain: \texttt{forcing\_term = dt * force\_contrib + M * U\_old}

\textbf{Usage:}
\begin{lstlisting}[language=Python, caption=Forcing Terms Usage]
# Compute forcing terms for implicit Euler step
current_bulk_data = [...]  # Current solution
forcing_terms = lean_manager.compute_forcing_terms(
    bulk_data_list=current_bulk_data,
    problems=problems,
    discretizations=discretizations,
    time=0.1,
    dt=0.01
)

# forcing_terms[i] has shape (2*neq, n_elements) for domain i
\end{lstlisting}

\subsection{Initialization Methods}
\label{subsec:initialization_methods}

\paragraph{initialize\_all\_bulk\_data()}\leavevmode
\begin{lstlisting}[language=Python, caption=Initialize All BulkData Method]
def initialize_all_bulk_data(self, 
                            problems: List,
                            discretizations: List,
                            time: float = 0.0) -> List[BulkData]
\end{lstlisting}

\textbf{Parameters:}
\begin{itemize}
    \item \texttt{problems}: List of Problem objects
    \item \texttt{discretizations}: List of discretization objects
    \item \texttt{time}: Initial time (default: 0.0)
\end{itemize}

\textbf{Returns:} \texttt{List[BulkData]} - Initialized BulkData objects containing primal representation of initial conditions for all domains; calls {\tt initialize\_bulk\_data\_from\_initial\_conditions} for all domains

\textbf{Usage:}
\begin{lstlisting}[language=Python, caption=Initialize All Usage]
# Initialize all domains with initial conditions
initial_bulk_data = lean_manager.initialize_all_bulk_data(
    problems=problems,
    discretizations=discretizations,
    time=0.0
)

print(f"Initialized {len(initial_bulk_data)} BulkData objects")
\end{lstlisting}

\paragraph{initialize\_bulk\_data\_from\_initial\_conditions()}\leavevmode
\begin{lstlisting}[language=Python, caption=Initialize Single Domain Method]
def initialize_bulk_data_from_initial_conditions(self, 
                                                domain_index: int,
                                                problem,
                                                discretization,
                                                time: float = 0.0) -> BulkData
\end{lstlisting}

\textbf{Parameters:}
\begin{itemize}
    \item \texttt{domain\_index}: Index of domain to initialize
    \item \texttt{problem}: Problem object for this domain
    \item \texttt{discretization}: Discretization object for this domain
    \item \texttt{time}: Initial time (default: 0.0)
\end{itemize}

\textbf{Returns:} \texttt{BulkData} - Initialized BulkData object containing primal representation of initial conditions

\textbf{Process:}
\begin{enumerate}
    \item Validate domain index and framework objects
    \item Create BulkData with primal formulation
    \item Set initial conditions from stored domain data
    \item Default to zero if no initial conditions available
\end{enumerate}

\textbf{Usage:}
\begin{lstlisting}[language=Python, caption=Initialize Single Domain Usage]
# Initialize specific domain
domain_0_bulk = lean_manager.initialize_bulk_data_from_initial_conditions(
    domain_index=0,
    problem=problems[0],
    discretization=discretizations[0],
    time=0.0
)
\end{lstlisting}

\subsection{Data Management Methods}
\label{subsec:data_management}

\paragraph{update\_bulk\_data()}\leavevmode
\begin{lstlisting}[language=Python, caption=Update BulkData Method]
def update_bulk_data(self, bulk_data_list: List[BulkData], new_data_list: List[np.ndarray])
\end{lstlisting}

\textbf{Parameters:}
\begin{itemize}
    \item \texttt{bulk\_data\_list}: List of BulkData objects to update
    \item \texttt{new\_data\_list}: List of new bulk solution arrays
\end{itemize}

\textbf{Returns:} \texttt{None}

\textbf{Side Effects:} Updates data in all BulkData objects

\textbf{Validation:}
\begin{itemize}
    \item Lists must have matching lengths
    \item New data must have compatible shapes
    \item New data must not contain NaN or infinite values
\end{itemize}

\textbf{Usage:}
\begin{lstlisting}[language=Python, caption=Update BulkData Usage]
# Update bulk data with new solution
new_solutions = [...]  # List of numpy arrays
lean_manager.update_bulk_data(
    bulk_data_list=current_bulk_data,
    new_data_list=new_solutions
)
\end{lstlisting}

\paragraph{get\_bulk\_data\_arrays()}\leavevmode
\begin{lstlisting}[language=Python, caption=Get Data Arrays Method]
def get_bulk_data_arrays(self, bulk_data_list: List[BulkData]) -> List[np.ndarray]
\end{lstlisting}

\textbf{Parameters:}
\begin{itemize}
    \item \texttt{bulk\_data\_list}: List of BulkData objects
\end{itemize}

\textbf{Returns:} \texttt{List[np.ndarray]} - Data arrays from all BulkData objects

\textbf{Usage:}
\begin{lstlisting}[language=Python, caption=Get Arrays Usage]
# Extract data arrays for external processing
data_arrays = lean_manager.get_bulk_data_arrays(bulk_data_list)
for i, array in enumerate(data_arrays):
    print(f"Domain {i} data shape: {array.shape}")
\end{lstlisting}

\subsection{Analysis Methods}
\label{subsec:analysis_methods}

\paragraph{compute\_total\_mass()}\leavevmode
\begin{lstlisting}[language=Python, caption=Compute Total Mass Method]
def compute_total_mass(self, bulk_data_list: List[BulkData]) -> float
\end{lstlisting}

\textbf{Parameters:}
\begin{itemize}
    \item \texttt{bulk\_data\_list}: List of BulkData instances
\end{itemize}

\textbf{Returns:} \texttt{float} - Total mass across all domains

\textbf{Computation:} Sums mass from each domain using stored mass matrices

\textbf{Usage:}
\begin{lstlisting}[language=Python, caption=Mass Computation Usage]
# Monitor mass conservation
initial_mass = lean_manager.compute_total_mass(initial_bulk_data)
current_mass = lean_manager.compute_total_mass(current_bulk_data)

mass_change = abs(current_mass - initial_mass) / initial_mass
print(f"Relative mass change: {mass_change:.6e}")
\end{lstlisting}




\paragraph{compute\_mass\_conservation()}\leavevmode
\begin{lstlisting}[language=Python, caption=Mass Conservation Method]
def compute_mass_conservation(self, bulk_data_list: List[BulkData]) -> float
\end{lstlisting}

\textbf{Parameters:}
\begin{itemize}
    \item \texttt{bulk\_data\_list}: List of BulkData instances
\end{itemize}

\textbf{Returns:} \texttt{float} - Total mass (alias for \texttt{compute\_total\_mass})

\textbf{Note:} This method is an alias for consistency with other interfaces

\begin{warningbox}
	Both methods compute the sum of the masses of the different fields. In our framework we should compute the individual mass of the individual fields.
\end{warningbox}

\subsection{Utility Methods}
\label{subsec:utility_methods}

\paragraph{get\_num\_domains()}\leavevmode
\begin{lstlisting}[language=Python, caption=Get Number of Domains Method]
def get_num_domains(self) -> int
\end{lstlisting}

\textbf{Returns:} \texttt{int} - Number of domains managed

\textbf{Usage:}
\begin{lstlisting}[language=Python, caption=Get Domains Count Usage]
n_domains = lean_manager.get_num_domains()
print(f"Managing {n_domains} domains")
\end{lstlisting}

\paragraph{get\_domain\_info()}\leavevmode
\begin{lstlisting}[language=Python, caption=Get Domain Info Method]
def get_domain_info(self, domain_idx: int) -> DomainData
\end{lstlisting}

\textbf{Parameters:}
\begin{itemize}
    \item \texttt{domain\_idx}: Domain index
\end{itemize}

\textbf{Returns:} \texttt{DomainData} - Domain data object for inspection

\textbf{Raises:} \texttt{IndexError} for invalid domain index

\textbf{Usage:}
\begin{lstlisting}[language=Python, caption=Domain Info Usage]
# Inspect domain properties
domain_info = lean_manager.get_domain_info(0)
print(f"Domain 0: {domain_info.neq} equations, {domain_info.n_elements} elements")
print(f"Element length: {domain_info.element_length}")
\end{lstlisting}

\subsection{Testing and Validation}
\label{subsec:testing_validation}

\paragraph{test()}\leavevmode
\begin{lstlisting}[language=Python, caption=Test Method]
def test(self, 
         problems: List = None,
         discretizations: List = None,
         static_condensations: List = None) -> bool
\end{lstlisting}

\textbf{Parameters:}
\begin{itemize}
    \item \texttt{problems}: List of Problem objects for testing (optional)
    \item \texttt{discretizations}: List of discretization objects for testing (optional)
    \item \texttt{static\_condensations}: List of static condensation objects for testing (optional)
\end{itemize}

\textbf{Returns:} \texttt{bool} - True if all tests pass, False otherwise

\textbf{Test Suite:}
\begin{enumerate}
    \item \textbf{Framework Object Validation Test}: Validates provided framework objects
    \item \textbf{Domain Data Structure Test}: Validates stored domain data integrity
    \item \textbf{BulkData Creation Test}: Tests creation of primal and dual BulkData
    \item \textbf{Initialization Test}: Tests bulk data initialization
    \item \textbf{Forcing Term Computation Test}: Tests forcing term calculations
    \item \textbf{Mass Computation Test}: Tests mass conservation calculations
    \item \textbf{Bounds Checking Test}: Tests error handling for invalid indices
    \item \textbf{Utility Methods Test}: Tests helper methods
    \item \textbf{Parameter Mismatch Test}: Tests validation error detection
\end{enumerate}

\textbf{Usage:}
\begin{lstlisting}[language=Python, caption=Test Method Usage]
# Comprehensive testing with framework objects
if lean_manager.test(
    problems=problems,
    discretizations=discretizations,
    static_condensations=static_condensations
):
    print("✓ Lean BulkDataManager is fully functional")
else:
    print("✗ Issues detected in Lean BulkDataManager")

# Minimal testing without framework objects
if lean_manager.test():
    print("✓ Basic structure validation passed")
\end{lstlisting}

\textbf{Sample Test Output:}
\begin{lstlisting}[language=Python, caption=Sample Test Output]
Testing Lean BulkDataManager with 3 domains
PASS: Framework object validation passed
PASS: All domain data validated
PASS: BulkData creation tests passed
PASS: Initialization tests passed
PASS: Forcing term computation tests passed
PASS: Mass computation test passed (total_mass=1.234567e+00)
PASS: ValueError raised for negative domain index
PASS: get_num_domains() returned correct value
PASS: get_domain_info() test passed
PASS: Correctly detected wrong number of problems
PASS: Correctly detected incompatible problem neq
PASS: Parameter mismatch detection tests passed
✓ All Lean BulkDataManager tests passed!
\end{lstlisting}

\subsection{Special Methods}
\label{subsec:special_methods}

\paragraph{\_\_str\_\_()}\leavevmode
\begin{lstlisting}[language=Python, caption=String Representation Method]
def __str__(self) -> str
\end{lstlisting}

\textbf{Returns:} \texttt{str} - Human-readable summary

\textbf{Format:} \texttt{"LeanBulkDataManager(domains=N, total\_elements=M, total\_equations=K)"}

\paragraph{\_\_repr\_\_()}\leavevmode
\begin{lstlisting}[language=Python, caption=Repr Method]
def __repr__(self) -> str
\end{lstlisting}

\textbf{Returns:} \texttt{str} - Developer-oriented representation

\textbf{Format:} \texttt{"LeanBulkDataManager(n\_domains=N, domain\_elements=[...], domain\_equations=[...])"}

\textbf{Usage:}
\begin{lstlisting}[language=Python, caption=String Methods Usage]
print(str(lean_manager))
# Output: LeanBulkDataManager(domains=3, total_elements=60, total_equations=6)

print(repr(lean_manager))
# Output: LeanBulkDataManager(n_domains=3, domain_elements=[20, 20, 20], domain_equations=[2, 2, 2])
\end{lstlisting}

\subsection{Complete Usage Examples}
\label{subsec:complete_usage_examples}

\subsubsection{Standard Workflow Example}

\begin{lstlisting}[language=Python, caption=Complete Lean Manager Workflow]
from bionetflux.core.lean_bulk_data_manager import BulkDataManager
from bionetflux.core.problem import Problem
from bionetflux.core.discretization import Discretization



# Step 1: Create framework objects (problems, discretizations, static_condensations)
# problems = [...]  # List of Problem instances
# discretizations = [...]  # List of Discretization instances  
# static_condensations = [...]  # List of static condensation instances
from setup_solver import quick_setup
filename = "bionetflux.problems.reduced_ooc_problem" 
setup = quick_setup(filename, validate=True)
problems = setup.problems
discretizations = setup.global_discretization.spatial_discretizations
static_condensations = setup.static_condensations

# Step 2: Extract essential data once (memory-efficient)
domain_data_list = BulkDataManager.extract_domain_data_list(
problems=problems,
discretizations=discretizations,
static_condensations=static_condensations
)

# Step 3: Create lean manager with extracted data only
lean_manager = BulkDataManager(domain_data_list)

# Step 4: Validate compatibility (optional but recommended)
if not lean_manager.test(problems, discretizations, static_condensations):
raise RuntimeError("Framework objects incompatible with extracted data")

# Step 5: Initialize bulk data for all domains
bulk_data_list = lean_manager.initialize_all_bulk_data(
problems=problems,
discretizations=discretizations,
time=0.0
)

# Step 6: Time evolution loop
dt = 0.01
for time_step in range(1):
current_time = time_step * dt

# Compute forcing terms for implicit Euler
forcing_terms = lean_manager.compute_forcing_terms(
bulk_data_list=bulk_data_list,
problems=problems,
discretizations=discretizations,
time=current_time,
dt=dt
)

# Initialize new primal bulk object with random entries

new_bulk_data_list = []
for i, bulk_data in enumerate(bulk_data_list):
new_bulk_primal = lean_manager.create_bulk_data(
domain_index=i,
problem=problems[i],
discretization=discretizations[i],
dual=False
)

# Set random data for the primal bulk object
domain_data = lean_manager.get_domain_info(i)
random_shape = (2 * domain_data.neq, domain_data.n_elements)
random_data = np.random.rand(*random_shape) * 0.1  # Small random values
new_bulk_primal.set_data(random_data)

new_bulk_data_list.append(new_bulk_primal)

# Create new_solutions list (placeholder for actual solver results)
new_data = [bulk_data.get_data() for bulk_data in new_bulk_data_list]

# Update bulk data with new solutions
lean_manager.update_bulk_data(bulk_data_list, new_data)

# Monitor mass conservation
current_mass = lean_manager.compute_total_mass(bulk_data_list)
if time_step % 10 == 0:
print(f"Time {current_time:.3f}: Mass = {current_mass:.6e}")

print("✓ Time evolution completed with lean manager")
\end{lstlisting}



\subsubsection{Multi-Manager Example}

\begin{lstlisting}[language=Python, caption=Multiple Lean Managers from Same Data]
# Extract domain data once
domain_data_list = BulkDataManager.extract_domain_data_list(
    problems, discretizations, static_condensations
)

# Create multiple lean managers for different purposes
# (all sharing the same extracted data - no additional memory cost)

# Manager for time evolution
evolution_manager = BulkDataManager(domain_data_list)

# Manager for forcing term computation
forcing_manager = BulkDataManager(domain_data_list)

# Manager for mass conservation tracking  
conservation_manager = BulkDataManager(domain_data_list)

# Each manager can operate independently but uses same base data
initial_bulk = evolution_manager.initialize_all_bulk_data(problems, discretizations)
source_terms = forcing_manager.compute_source_terms(problems, discretizations, time=0.0)
total_mass = conservation_manager.compute_total_mass(initial_bulk)

print(f"Created 3 independent managers sharing {len(domain_data_list)} domain data objects")
\end{lstlisting}

\subsection{Method Summary Table}
\label{subsec:lean_method_summary}

\begin{longtable}{|p{5.3cm}|p{3.2cm}|p{5cm}|}
\hline
\textbf{Method} & \textbf{Returns} & \textbf{Purpose} \\
\hline
\endhead

\texttt{\_\_init\_\_} & \texttt{None} & Initialize with extracted domain data only \\
\hline

\texttt{extract\_domain\_data\_list} & \texttt{List[DomainData]} & Static factory for one-time data extraction \\
\hline

\texttt{create\_bulk\_data} & \texttt{BulkData} & Create BulkData using external framework objects \\
\hline

\texttt{initialize\_all\_bulk\_data} & \texttt{List[BulkData]} & Initialize all domains with initial conditions \\
\hline

\texttt{compute\_source\_terms} & \texttt{List[BulkData]} & Compute source terms using dual formulation \\
\hline

\texttt{compute\_forcing\_terms} & \texttt{List[np.ndarray]} & Compute forcing terms for implicit Euler \\
\hline

\texttt{update\_bulk\_data} & \texttt{None} & Update BulkData objects with new solutions \\
\hline

\texttt{compute\_total\_mass} & \texttt{float} & Calculate total mass for conservation \\
\hline

\texttt{get\_bulk\_data\_arrays} & \texttt{List[np.ndarray]} & Extract data arrays from BulkData objects \\
\hline

\texttt{get\_num\_domains} & \texttt{int} & Get number of managed domains \\
\hline

\texttt{get\_domain\_info} & \texttt{DomainData} & Access domain data for inspection \\
\hline

\texttt{test} & \texttt{bool} & Comprehensive validation and testing \\
\hline

\texttt{\_validate\_framework\_objects} & \texttt{None} & Validate framework object compatibility \\
\hline

\end{longtable}

This documentation provides an exact reference for the lean BulkDataManager class, emphasizing its memory-efficient design and parameter-based approach to framework object usage. The lean architecture minimizes memory overhead while maintaining full functionality through external object validation and flexible method interfaces.

% End of lean bulk data manager module API documentation

\begin{todobox}
	\begin{itemize}
		\item Rename module file as {\tt bulk\_data\_manager} or possibly also {\tt data\_manager}
		\item Evaluate moving (or duplicating) {\tt extract\_domain\_data\_list} to {\tt DomainData}
		\item Fix the mass computation and mass monitoring so that they compute the mass of individual fields
		\item The examples should use a toy quick-setup, now they use an ooc type problem
	\end{itemize}
\end{todobox}



\input{static_condensation_modules_api}
\input{flux_jump_module_api}


\input{lean_global_assembly_api}
\input{elementary_matrices_module_api}


\input{setup_solver_detailed_api}

\input{lean_matplotlib_plotter_api}

\input{problems_folder_detailed_api}

\section{Example Applications}

\subsection{Example 1: Simple Keller-Segel Chain}

\begin{lstlisting}[language=Python, caption={Simple Keller-Segel Example}]
# File: examples/simple_keller_segel.py
import sys
sys.path.insert(0, '../code')

from setup_solver import quick_setup
from ooc1d.visualization.lean_matplotlib_plotter import LeanMatplotlibPlotter

def main():
    # Setup problem
    setup = quick_setup("ooc1d.problems.KS_with_geometry", validate=True)
    
    # Get initial conditions
    trace_solutions, multipliers = setup.create_initial_conditions()
    
    # Initialize plotter
    plotter = LeanMatplotlibPlotter(
        problems=setup.problems,
        discretizations=setup.global_discretization.spatial_discretizations
    )
    
    # Plot initial state
    plotter.plot_2d_curves(trace_solutions, title="Initial State")
    plotter.plot_birdview(trace_solutions, equation_idx=0, time=0.0)
    
    # Time evolution
    dt = setup.global_discretization.dt
    T = 0.5
    current_time = 0.0
    global_solution = setup.create_global_solution_vector(
        trace_solutions, multipliers)
    
    while current_time < T:
        # Newton iteration (simplified)
        current_time += dt
        # ... solver steps ...
        
        # Extract solutions
        final_traces, _ = setup.extract_domain_solutions(global_solution)
        
        # Visualize
        plotter.plot_birdview(final_traces, equation_idx=0, 
                             time=current_time)
    
    plotter.show_all()

if __name__ == "__main__":
    main()
\end{lstlisting}

\subsection{Example 2: Complex Grid Network}

\begin{lstlisting}[language=Python, caption={Grid Network Example}]
# File: examples/grid_network_example.py
import sys
sys.path.insert(0, '../code')

from setup_solver import quick_setup
from ooc1d.visualization.lean_matplotlib_plotter import LeanMatplotlibPlotter

def main():
    # Load complex grid problem
    setup = quick_setup("ooc1d.problems.KS_grid_geometry", validate=True)
    
    print(f"Problem: {setup.get_problem_info()['problem_name']}")
    print(f"Domains: {setup.get_problem_info()['num_domains']}")
    
    # Initial conditions
    trace_solutions, multipliers = setup.create_initial_conditions()
    
    # Visualization
    plotter = LeanMatplotlibPlotter(
        problems=setup.problems,
        discretizations=setup.global_discretization.spatial_discretizations,
        figsize=(15, 10)
    )
    
    # Multiple views of initial state
    plotter.plot_2d_curves(
        trace_solutions, 
        title="Grid Network - Domain Profiles",
        save_filename="grid_profiles.png"
    )
    
    for eq_idx in range(2):  # Both equations
        plotter.plot_flat_3d(
            trace_solutions,
            equation_idx=eq_idx,
            title=f"Grid Network - {plotter.equation_names[eq_idx]} (3D)",
            save_filename=f"grid_3d_eq{eq_idx}.png"
        )
        
        plotter.plot_birdview(
            trace_solutions,
            equation_idx=eq_idx,
            time=0.0,
            save_filename=f"grid_birdview_eq{eq_idx}.png"
        )
    
    plotter.show_all()

if __name__ == "__main__":
    main()
\end{lstlisting}

\section{API Reference}

\subsection{Quick Setup Function}

\begin{lstlisting}[language=Python, caption={Quick Setup API}]
setup_solver.quick_setup(problem_module: str, 
                         validate: bool = True) -> SolverSetup
\end{lstlisting}

\textbf{Parameters:}
\begin{itemize}
    \item \code{problem\_module}: Import path to problem definition (e.g., "ooc1d.problems.my\_problem")
    \item \code{validate}: Whether to validate setup after creation
\end{itemize}

\textbf{Returns:} Configured \code{SolverSetup} object

\subsection{SolverSetup Class}

\begin{lstlisting}[language=Python, caption={SolverSetup API}]
class SolverSetup:
    def get_problem_info() -> Dict[str, Any]
    def create_initial_conditions() -> Tuple[List[np.ndarray], np.ndarray]
    def create_global_solution_vector(traces, multipliers) -> np.ndarray
    def extract_domain_solutions(global_solution) -> Tuple[List[np.ndarray], 
                                                           np.ndarray]
\end{lstlisting}

\subsection{DomainGeometry Class}

\begin{lstlisting}[language=Python, caption={DomainGeometry API}]
class DomainGeometry:
    def add_domain(extrema_start: Tuple[float, float],
                   extrema_end: Tuple[float, float],
                   domain_start: float = None,
                   domain_length: float = None,
                   name: str = None,
                   **metadata) -> int
    
    def get_domain(domain_id: int) -> DomainInfo
    def get_bounding_box() -> Dict[str, float]
    def num_domains() -> int
    def summary() -> str
\end{lstlisting}

\subsection{LeanMatplotlibPlotter Class}

\begin{lstlisting}[language=Python, caption={Plotter API}]
class LeanMatplotlibPlotter:
    def __init__(problems, discretizations, 
                 equation_names=None, figsize=(12,8))
    
    def plot_2d_curves(trace_solutions, title, 
                       show_mesh_points=True,
                       save_filename=None) -> plt.Figure
    
    def plot_flat_3d(trace_solutions, equation_idx=0, 
                     view_angle=(30,45),
                     save_filename=None) -> plt.Figure
    
    def plot_birdview(trace_solutions, equation_idx=0, 
                      time=0.0,
                      save_filename=None) -> plt.Figure
    
    def plot_comparison(initial_traces, final_traces, 
                        initial_time=0.0,
                        final_time=1.0, 
                        save_filename=None) -> plt.Figure
\end{lstlisting}



\input{project_status_report}
\input{project_todo_analysis}

\section{Troubleshooting}

\subsection{Common Issues}

\subsubsection{Import Errors}
\begin{lstlisting}[language=Python, caption={Path Setup}]
# Ensure correct path setup
import sys
sys.path.insert(0, '/path/to/BioNetFlux/code')
\end{lstlisting}

\subsubsection{Geometry Validation}
\begin{lstlisting}[language=Python, caption={Geometry Debugging}]
# Check geometry before problem creation
geometry = DomainGeometry("test")
# ... add domains ...
print(geometry.summary())  # Verify domain layout
print(geometry.get_bounding_box())  # Check coordinates
\end{lstlisting}

\subsubsection{Constraint Setup}
\begin{lstlisting}[language=Python, caption={Constraint Verification}]
# Verify constraint mapping
constraint_manager.map_to_discretizations(discretizations)
print(f"Total constraints: {constraint_manager.n_multipliers}")
\end{lstlisting}

\subsubsection{Solution Convergence}
\begin{lstlisting}[language=Python, caption={Convergence Monitoring}]
# Monitor Newton iteration
newton_tolerance = 1e-10
max_newton_iterations = 20

# Check residual norms during iteration
if residual_norm > newton_tolerance:
    print(f"Convergence issue: residual = {residual_norm:.2e}")
\end{lstlisting}

\subsection{Performance Optimization}

\begin{enumerate}
    \item \textbf{Mesh Resolution}: Balance accuracy vs. computational cost
    \item \textbf{Time Step Size}: Use adaptive time stepping for stability
    \item \textbf{Newton Tolerance}: Adjust based on problem requirements
    \item \textbf{Domain Decomposition}: Optimize domain sizes for load balancing
\end{enumerate}

\subsection{Debugging Tips}

\begin{enumerate}
    \item \textbf{Visualization}: Use all three plot types to understand solution behavior
    \item \textbf{Parameter Validation}: Check physical parameter ranges
    \item \textbf{Constraint Verification}: Ensure proper interface connectivity
    \item \textbf{Solution Monitoring}: Track solution norms and residuals
\end{enumerate}

\section{Contact and Support}

For questions, issues, or contributions:

\begin{itemize}
    \item \textbf{Repository}: [\bionetflux{} GitHub]
    \item \textbf{Documentation}: See \code{docs/} directory
    \item \textbf{Examples}: See \code{examples/} directory
    \item \textbf{Issues}: Submit via GitHub Issues
\end{itemize}

\vspace{2cm}

\begin{center}
\textbf{\bionetflux{} Development Team} \\
\textit{Multi-Domain Biological Network Flow Simulation Framework}
\end{center}

\end{document}
